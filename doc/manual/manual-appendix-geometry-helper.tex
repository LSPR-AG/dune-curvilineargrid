\subsection{CurvilinearGeometryHelper}
\label{interface-geometry-helper}

\noindent
This section will discuss the interface of the \textit{CurvilinearGeometryHelper} class. This is an auxiliary class of \curvgeom{}, which provides functionality for addressing subentities of a uniform interpolatory grid, which are later used by \curvgeom{} to address the subentity geometries. \\


\noindent
Hard-coded number of curvilinear degrees of freedom as given in \ref{theory-lagrange}

\begin{mybox}
\begin{lstlisting}
  static int dofPerOrder(Dune::GeometryType geomType, int order)
\end{lstlisting}
\end{mybox}

\noindent
Hard-coded map between a corner internal index and vertex internal index.

\begin{mybox}
\begin{lstlisting}
  static InternalIndexType cornerIndex(Dune::GeometryType geomType, int order, InternalIndexType i)
\end{lstlisting}
\end{mybox}

\noindent
A procedure to extract corner indices from a vertex index set % using the $cornerIndex$ command

\begin{mybox}
\begin{lstlisting}
  template<class ct, int mydim>
  static std::vector<int> entityVertexCornerSubset(Dune::GeometryType gt, const std::vector<int> & vertexIndexSet, InterpolatoryOrderType order)
\end{lstlisting}
\end{mybox}

\noindent
A procedure to find the coordinate of a corner of a reference element based on its index. %Should be replaced by $ref.position()$ together with other methods of this class.

\begin{mybox}
\begin{lstlisting}
  template <typename ctype, int cdim>
  static Dune::FieldVector<ctype, cdim> cornerInternalCoordinate(GeometryType gt, InternalIndexType subInd)
\end{lstlisting}
\end{mybox}

\noindent
A procedure to generate a set of integer coordinates for the uniform interpolatory simplex grid (see \cref{fig:lagrange:enumerationconstruction})

\begin{mybox}
\begin{lstlisting}
  template <int mydim>
  static IntegerCoordinateVector simplexGridEnumerate(int n)
\end{lstlisting}
\end{mybox}

\noindent
A procedure to generate the local vertex set for a uniform interpolatory simplex grid (see \cref{fig:lagrange:enumerationconstruction}). Can re-use the already existing \textit{simplexGridEnumerate} array for speedup

\begin{mybox}
\begin{lstlisting}
  template <class ct, int mydim>
  static std::vector<Dune::FieldVector<ct, mydim> > simplexGridCoordinateSet(int n)
  template <class ct, int mydim>
  static std::vector<Dune::FieldVector<ct, mydim> > simplexGridCoordinateSet(IntegerCoordinateVector integerGrid, int n)
\end{lstlisting}
\end{mybox}


\noindent
A procedure to extract the local interpolatory vertex grid of a subentity of a given entity. \textit{subentityCodim} determines the codimension of the subentity, \textit{subentityIndex} determines the internal index of the subentity within the parent entity.

\begin{mybox}
\begin{lstlisting}
    template <class ct, int cdim>
    static std::vector<InternalIndexType> subentityInternalCoordinateSet(Dune::GeometryType entityGeometry, int order, int subentityCodim, int subentityIndex)
\end{lstlisting}
\end{mybox}

