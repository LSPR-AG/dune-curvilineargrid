\section{Duffy Transform}
\label{section-abstract-duffy-transform}

The Duffy transform is is a bijective map between the local coordinate of a hypercube and another entity, in our case, the reference simplex. Further, one can use the Duffy transform to directly apply the hypercubic integration routines (e.g. tensor product quadrature rules) to simplex domains. Effectively, Duffy transform claims that $ \int_{\triangle} f(\vec{x}) d\vec{x} = \int_{\Box} g(\vec{\tau}) f(\vec{x}(\vec{\tau})) d\vec{\tau}$, giving explicit expressions for $g(\vec{\tau})$ and $\vec{x}(\vec{\tau})$. \\

\noindent
We consider simplices of all dimensions up to 3: \\

\noindent
\textbf{1D}: The reference hypercube and simplex are both an edge defined on $[0, 1]$, so they have exactly the same parameter. \\

\noindent
\textbf{2D}: Here we map from reference triangle to reference square. We transform the integral of interest, namely

\[ \int_0^1 \int_0^{1-x} f \biggl(
\begin{matrix}  x \\ y  \end{matrix} \biggr)
 dx dy = \int_0^1 \int_0^1 (1-x) 
f \biggl( \begin{matrix} x \\ (1-x)t \end{matrix} \biggr)
dx dt \]

\noindent
using a substitution $t = y / (1 - x)$, such that $t \in [0, 1]$. \\

\noindent
\textbf{3D}: Here we map from reference tetrahedron to reference cube. We transform the integral of interest, namely

\[ \int_0^1 \int_0^{1-x} \int_0^{1-x-y} f \scalebox{1.2}{\Bigg(}
\begin{matrix}  x \\ y \\ z  \end{matrix} \scalebox{1.2}{\Bigg)}
dx dy = \int_0^1 \int_0^1 \int_0^1 (1-x)^2 (1-t)
f \scalebox{1.2}{\Bigg(} \begin{matrix}  x \\ (1-x)t \\ (1-x)(1-t)\tau  \end{matrix} \scalebox{1.2}{\Bigg)}
dx dt d\tau \]

\noindent
using a substitution $t = \frac{y}{1 - x}, \tau = \frac{z}{(1 - x)(1 - t)}$, such that $t,\tau \in [0, 1]$. \\