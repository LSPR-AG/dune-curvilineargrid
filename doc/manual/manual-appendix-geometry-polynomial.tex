\subsection{Polynomial Class}
\label{interface-geometry-polynomial}

\noindent
An arbitrary polynomial of order $n$ with $d$ parameters can be represented in its expanded form as
\[ p(\vec{u}) = \sum_i A_i \prod_{j = 0}^d u_j^{\mathrm{pow}_{i,j}},  \]
where ${pow}_{i,j}$ is the power $j$\textsuperscript{th} dimension of $i$\textsuperscript{th} summand. For example, in 3D this can be written as
\[ p(\vec{u}) = \sum_i A_i u^{pow_{u,i}} v^{pow_{v,i}} w^{pow_{w, i}},  \]

\noindent
We define a Monomial class, which stores a constant multiplier $A$ and vector of powers $pow$.

\begin{mybox}
\begin{lstlisting}
  PolynomialTraits::Monomial(double prefNew, std::vector<int> powerNew)
\end{lstlisting}
\end{mybox}

\noindent
A polynomial can be constructed either empty, from a single monomial or from another polynomial.

\begin{mybox}
\begin{lstlisting}
  Polynomial()
  Polynomial(Monomial M)
  Polynomial(const Polynomial & other)
\end{lstlisting}
\end{mybox}

\noindent
The below interface provides methods to perform basic algebraic operations with polynomials and scalars. The method $axpy$ is the scaled addition, equivalent to $this += other * a$

\begin{mybox}
\begin{lstlisting}
  LocalPolynomial & operator+=(const Monomial & otherM)
  LocalPolynomial & operator+=(const LocalPolynomial & other)
  LocalPolynomial & operator*=(const double c)  
  LocalPolynomial & operator*=(const LocalPolynomial & other)  
  void axpy(LocalPolynomial other, double c)
  LocalPolynomial operator+(const LocalPolynomial & other)
  LocalPolynomial operator+(const ctype a)  
  LocalPolynomial operator-(const LocalPolynomial & other)  
  LocalPolynomial operator-(const ctype a)  
  LocalPolynomial operator*(const ctype a)  
  LocalPolynomial operator*(const LocalPolynomial & other)
\end{lstlisting}
\end{mybox}

\noindent
We have implemented differentiation, integration and evaluation a polynomials. $derivative$ routine returns the partial derivative of a polynomial w.r.t. coordinate indexed by the parameter; $evaluate$ routine evaluates the polynomial at the provided local coordinate; $integrateRefSimplex$ routine integrates the polynomial over the reference entity of the same dimension as the polynomial, returning a scalar. An integral of a monomial over the reference simplex has an analytical expression, see \ref{appendix-proof-simplexintegral}

\begin{mybox}
\begin{lstlisting}
  LocalPolynomial derivative(int iDim)
  double evaluate(const LocalCoordinate & point)
  double integrateRefSimplex()
\end{lstlisting}
\end{mybox}

\noindent
The following auxiliary methods can be used to provide additional information about a polynomial. $order$ routine returns the largest power among all monomials, that is, the sum of powers of that monomial;
$magnitude$ routine returns the largest absolute value prefactor over all monomials. $to\_string$ routine converts the polynomial to a string for further text output

\begin{mybox}
\begin{lstlisting}
  unsigned int order()
  double magnitude()
  std::string to_string()
\end{lstlisting}
\end{mybox}

\noindent
Also, caching is implemented via the $cache()$ routine. This method can be called after the polynomial will no longer be changed, but only evaluated. Pre-computing factorials and monomial powers accelerates further evaluation and analytical integration of the polynomial.


%\noindent
%\uline{Compactify}: adds up all summands with the same power. Sorts the summands by $(x_1,y_1,z_1) < (x_2, y_2, z_2)$, where $x$ has the highest priority and $z$ has the lowest priority. Then all of the repeating powers will be consecutive. Simply loop over sorted polynomial, and to a new polynomial add the sums of all consecutive repeating polynomials.



%%\subsection{Tests}
%%
%%
%%\noindent
%%Currently the tests are only for 1, 2 and 3 dimensions. Most of the tests use intrinsic functionality like polynomial operators and derivatives to construct polynomials and print them to the screen, and request the user to to verify manually if they match the expected polynomials which are also printed. For each dimension there is one test which integrates a non-linear polynomial over simplex and prints out the result which is also compared manually. \\
%%
%%\textbf{TODO:} These tests can and should be automatized in the future using integer string comparison. The test program should throw an error if a test fails