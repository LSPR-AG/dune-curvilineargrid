\section{Singularity Elimination}

When implementing surface integral methods, one has to deal with singular integrals over the face of the element, which take the form
\[ \iint \frac{g(x,y)}{|\vec{x} - \vec{x}_0|} dx dy \]
where $g(x,y)$ is analytic. Principal value integration has to be used to analytically determine the value of the integral within the infinitesimal area patch including the singularity.
This is beyond the scope of the current section. Instead, we assume that the contribution of the singularity has already been considered. What remains is to develop a procedure to
numerically compute the remainder of the integral, eliminating the singularity and avoiding numerical instabilities. \\

\noindent
\textbf{[CITEDUFFY]} propose a method to eliminate the singularity for integrals over a reference triangle. Consider the integral over a reference triangle
\[ \int_0^1 \int_0^{1-x} \frac{g(x,y)}{|\vec{x} - \vec{x}_0|} dx dy \]
It is proposed to first deal with the special case $\vec{x}_0 = (1,0)$, that is, having the singularity in the bottom-right corner of the reference triangle.
Thus, the integral can be simplified to
\[ \int_0^1 \int_0^{1-x} \frac{g(x,y)}{\sqrt{(x-1)^2 + y^2}} dx dy \]
By changing the variable of the $y$-integral to $y=(1-x)t$, we get
\[ \int_0^1 \int_0^1 \frac{g(x,(1-x)t)}{\sqrt{(x-1)^2 + t^2(1-x)^2}} (1-x)dx dt \]
Since the contribution of the singularity itself is assumed to have already been accounted for, we may divide the numerator and denominator by $(1-x)$, ignoring the case $0/0$.
Thus, we have transformed an integral over a reference triangle to equivalent integral over a reference square
\[ \int_0^1 \int_0^1 \frac{g(x,(1-x)t)}{\sqrt{1 + t^2}} dx dt \]
which does not have a singularity, and can thus easily be integrated using your favourite quadrature.

\noindent
Now we need to generalize the above method to deal with all the singularities except $\vec{x}_0 = (1,0)$. This can be accomplished by subdividing the existing triangle
into (at most) 3 smaller triangles, for each of which the singularity is at the desired corner \textbf{[PICHERE]}. Thus, for each triangle we will construct a linear
mapping $(u,v) = \psi(u', v')$, where $(u', v')$ are the local coordinates of the sub-triangle. \\

\noindent
The problem is that the coordinate transformation slightly changes the form of the denominator, requiring us to adapt our procedure. In addition, the singularity is frequently given
in terms of the global coordinates, obtained, in general, using the curvilinear mapping $\vec{x} = \vec{p}(u, v)$. Considering both of these generalizations, we can write the more general
form of the singular integral as 
\[ \int_0^1 \int_0^{1-u'} \frac{g(u',v')}{|\vec{p}(\vec{\psi}(\vec{r}')) - \vec{p}(\vec{\psi}(\vec{r}_0'))|} du' dv' \]
where $\vec{r}' = (u', v')$ are the local coordinates of the sub-triangle. To simplify this equation, we can design a general polynomial mapping $p' = p \circ \psi$.
Since both maps are polynomial, the resulting map is obtained by substituting one polynomial into another.

\noindent
We use the same Duffy transform $v' = (1-u')t$ to obtain an integral over the reference square
\[ \int_0^1 \int_0^1 \frac{g(u',(1-u')t)}{|\vec{p}'(u',(1-u')t) - \vec{p}'(\vec{r}_0')|} (1-u')du' dt \]
Now let us consider, what happens to the denominator, if we choose the singularity to still be at the corner $\vec{r}_0' = (1,0)$.
The denominator $|\vec{p}'(u',(1-u')t) - \vec{p}'(1,0)|$, which is a polynomial, is clearly divisible by $(1-u')$, because $u' = 1$ is its solution. Again, we will divide the denominator
and numerator by $(1-u')$, eliminating the singularity, and obtaining the integral
\[ \int_0^1 \int_0^1 \frac{g(u',(1-u')t)}{|\vec{\zeta}(u', t)|} du' dt \]
All that remains is to find the explicit form of $\vec{\zeta}(u', t)$. By looping over all monomial summands of $\vec{p}'(u',v')$, we can rewrite it by splitting it into terms that
are proportional to $v'$ and terms that are not. For all the terms that are proportional to $v'$, we will bring one instance of $v'$ in front of the sum
\[ \vec{p}' = \vec{p}'(u', 0) + v' R(u', v') = \vec{p}'(u', 0) + (1-u')t R(u', v') \]
Thus, after division by $(1-u')$, we get
\[ \vec{\zeta} = \frac{\vec{p}'(u', 0) - \vec{p}'(1, 0)}{1-u'} + t R(u', v') \]
Finally, the division in the first fraction can be performed by considering that the numerator is a 1D polynomial of the form $\sum_{i \geq 1} a_i (x^i - 1)$.
Each monomial separately can be divided by $(1-u')$ by considering the identity
\[ \frac{x^n - 1}{x - 1} = \sum_{i=0}^{n-1} x^i \]


\textbf{[TODO]}
\begin{itemize}
  \item Show that resulting polynomial integral is not singular
  \item What to do with integrals over neighbouring elements
\end{itemize}