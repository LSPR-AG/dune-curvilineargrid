\section{Quadrature Integrator Class}
\label{interface-integrator-quadrature}

The Quadrature Integrator class provides static members to integrate functors over simplex reference elements by wrapping the quadrature rules provided by Dune. At the moment the dune-default Gauss-Legendre quadrature is used, however, the interface can be easily extended with flexibility to select the desired quadrature if need arises. The integrand functor must provide the operator \\

\begin{mybox}
\begin{lstlisting}
  double operator()(const Dune::FieldVector<ctype, mydim> & x)
\end{lstlisting}
\end{mybox}

\noindent
where $mydim$ must be equal to $gt.dim()$. Below 3 functions provide different strategies to approach integration \\

\begin{mybox}
\begin{lstlisting}
  template<class Functor>
  static ctype integrate(Dune::GeometryType gt, Functor f, int integrOrder)
  template<class Functor>
  static StatInfoVec integrateStat(Dune::GeometryType gt, Functor f, int integrOrderMax)
  template<class Functor>
  static StatInfo integrateRecursive(Dune::GeometryType gt, Functor f, ctype rel_tol)
\end{lstlisting}
\end{mybox}

\noindent
The first method integrates functor over the reference element using quadrature of a given order. The second method repeats the first method for all quadrature orders from 1 to specified maximal order, and outputs a vector of pairs (order, result). The last method integrates the functor with gradually increasing order until the desired estimated error tolerance is reached, or the maximal allowed quadrature order is reached. The estimated relative order is calculated as $\bigl | 1 - \frac{I(o)}{I_{smooth}(o-1)}  \bigr |$, where the smooth integral is defined as $I_{smooth}(o) = \zeta (I(o+1) + I(o-1)) + (1 - 2\zeta)I(o)$, with $\zeta$ being the smoothing parameter with default value $\zeta = 0.15$. The basic idea is to estimate the error as the difference between the integrals at consecutive orders.
\begin{itemize}
	\item Improvement 1: For some consecutive orders in Dune the number of quadrature points does not change, in this case simply skip to the next order.
	\item Improvement 2: Sometimes by random chance the two consecutive orders give very close results, while the integral still has not converged. To partially avoid this scenario, the smoothing parameter is introduced. The idea is to represent the previous best guess as a weighted average between a few previous estimates. So, if the integral has made a sharp jump recently, we will doubt that the integral has converged, thus the expected error will be larger.
\end{itemize}

\textbf{[TODO]}: To accelerate the gradual approximation strategy, it would be useful to implement a hierarchic quadrature. Such quadrature would double the interpolation order at every step, and reuse all the interpolation points already calculated.
