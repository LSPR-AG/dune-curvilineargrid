\section{Outline}
\label{section-outline}

\subsection{Capabilities}
\label{section-outline-capabilities}

The Curvilinear Grid (CurvGrid) is a self-consistent grid manager supporting 3D tetrahedral curvilinear grids. It depends on the core modules of Dune \citeDune{}, as well as an external parallel mesh partition library Parmetis \citeParMetis{}. Also, the CurvGrid depends on Curvilinear Geometry (CurvGeom), also developed by us as a separate dune-module. \\

\noindent
CurvGeom is capable of managing curvilinear simplex elements of orders 1-5 via hard-coded Lagrange Polynomials. It also has the capability to manage arbitrary order simplex geometries via analytic Largange Interpolation method. CurvGeom complies with standard dune-geometry interface, providing methods for local-to-global and global-to-local coordinate conversion, computation of Jacobian matrix, integration element and volume. There is also a cached version of CurvGeom which pre-computes the local-to-global map and its determinant, and is thus considerably faster than its non-cached analogue. Additionally, CurvGeom provides the methods to obtain outer normals of subentities of this geometry, and the subentity geometries themselves. Also, CurvGeom implements a polynomial class and associated differential methods, which allow to obtain analytical expressions for local-to-global map and associated integration element, enabling exact manipulation of geometries of arbitrary order. Finally, CurvGeom contains its own recursive integration tool (based on quadrature provided by dune-geometry) for integrating non-polynomial integrands. The reason for implementing this functionality is due to integration elements being non-polynomial in general case (see \textbf{CHAPTER ON INTEGRATION}). For example, the face area of a curvilinear element is given as an integral over a square root of a polynomial. \\

\noindent
CurvGrid is supplied with its own Curvilinear GMSH Reader (CurvReader). CurvReader is designed to read curvilinear \textit{.msh} files of orders 1 to 5, where 1 corresponds to linear meshes. CurvReader is parallel-scalable meaning that each process only reads the necessary part of the mesh, avoiding memory overflow. CurvReader has the option to partition the mesh using ParMetis during the reading procedure, further decreasing the file access time. CurvReader also reads physical tags provided GMSH and supplies the Grid Factory with this information. \\

\noindent
Curvgrid is also supplied with its own Curvilinear VTK Writer (CurvWriter) to output the grid into VTK, VTU or PVTU file formats. CurvWriter will write either the entire grid, or the set of elements supplied by user. It provides each element with data about its rank, partition type and physical tag, which are very useful for visualisation of the grid \textbf{[BULLSEYE PICS HERE]}. The mesh can be supplied with arbitrary number of vector fields to visualise, e.g. the solution(s) of a PDE. We have tested the visualisation of CurvGrid using \textbf{[PARAVIEW, VISIT]} \\

\noindent
Among other features of CurvGrid are: Global Index for all entities, Ghost Elements, DataHandle communication for entities of all codimensions, Nearest neighbour communication, Logging and Timing mechanisms, Curvilinear Grid diagnostics. A set of tutorial programs is provided to demonstrate the usage of various features of the grid. \\

\noindent
Below we list the functionality that is currently NOT available in the CurvGrid, but would be desirable to have. We welcome contributions from the community \\

\noindent
\begin{tabularx}{\textwidth}{l X}
\hline
     Difficulty & Description \\
\hline
	 Easy & Using GMSH partition tags to read pre-partitioned meshes \\
	 Medium & All-to-all communication paradigm to enable modelling of dense problems (e.g. Surface Integral Equation \textbf{[CITE OLIVIER HERE]}) \\
	 Medium & Curvilinear Grid optimization for parallel scalability using nearest neighbour communication \\
	 Medium & Curvilinear Meshes of non-uniform order \\
	 Medium & Utility to efficiently locate the element containing a given global coordinate (via OCTree) \\
	 Medium & Periodic and interior boundaries \\
	 Medium & 1D and 2D geometries \\
	 Medium & Does NOT support front/overlap elements at the moment \\
	 Hard & Non-tetrahedral geometries (e.g. Hexahedral) \\
	 Hard & Non-conformal meshes (Hanging nodes) \\
	 Hard & Global and Local Refinement \\
\hline
\end{tabularx}

\subsection{Design decisions}
\label{section-outline-designdecisions}

\begin{itemize}
	\item User must provide globalId's for vertices and elements. [Automatically implemented by GMSH \citeGMSH]
	\item User must provide all boundary segments inside GMSH file.
\end{itemize}


\subsection{Internal Structure}
\label{section-outline-internalstructure}

Below we present a simplified structure of classes used by Curvilinear Grid and geometry \\

\noindent
\begin{tabularx}{\textwidth}{ l | X }
\hline
   Class Name & Description \\ \hline
   CurvilinearGeometry                & Core class complying with dune-geometry standard \\ \hline
   * CurvilinearGeometryHelper        & Auxiliary functions, subentities for curvilinear elements \\ \hline
   * LagrangeInterpolator             & Lagrange Interpolation of element geometry \\ \hline
   * Polynomial                       & Analytic polynomial routines \\ \hline
   * DifferentialHelper               & Analytic Curvilinear Jacobians and Integration elements \\ \hline
   * IntegralHelper                   & Integral Functors, analytic integration routines  \\ \hline
   * QuadratureIntegrator             & Recursive integration using numerical quadrature. Performs fast \\ \hline
   * AdaptiveIntegrator               & Recursive integration using adaptive refinement. Performs slowly \\ \hline
  CurvilinearGMSHReader               & Reads a $.msh$ file and supplies entities to a Grid Factory \\ \hline
  * Gmsh2DuneMapper                   & Converts between interpolation vertex numbering paradigms \\ \hline
  CurvilinearVTKWriter                & Writes curvilinear elements to VTK, VTU/PVTU files \\ \hline
  CurvilinearVTKGridWriter            & Writes Curvilinear Grid to VTK, VTU/PVTU files \\ \hline
  CurvilinearGridBase                 & Lower level grid implementation. All entities are given by their codimension and index \\ \hline
  * CurvilinearGridStorage            & Storage class for entire grid \\ \hline
  * CurvilinearGridConstructor        & Constructs entities of the grid, finds neighbouring processes, generates global index \\ \hline
  * CurvilinearGhostConstructor       & Constructs ghost entities \\ \hline
  * CurvilinearPostConstructor        & Enables DataHandle communication \\ \hline
  Curvilinear Grid                    & Core class complying with dune-grid standard \\ \hline
  * CurvilinearGridFactory            & Interface for constructing Curvilinear Grid \\ \hline
  * CurvilinearGridDiagnostics        & Tests and collects statistics on Curvilinear Grid \\ \hline
  * LoggingMessage                    & Logging output with varying levels of verbosity \\ \hline
  * LoggingTimer                      & Parallel timing of parts of code. \\ \hline
  * AllCommunicate                    & Nearest neighbor communication for POD. Wrapper for $MPI\_alltoallv$ \\ \hline
  * VectorHelper                      & Manipulation of vectors (union, intersection, complement), as well as conversion to string \\ \hline
\end{tabularx}
