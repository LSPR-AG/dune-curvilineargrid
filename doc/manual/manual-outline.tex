\section{Outline}
\label{section-outline}

\subsection{Capabilities}
\label{section-outline-capabilities}

\noindent
Currently the curvilinear grid supports the following functionality.
\begin{itemize}
	\item Self-consistent grid manager supporting 3D tetrahedral curvilinear grids. Depends on Dune and Parmetis only.
	\item GMSH input files of curvilinear orders 1-5. Usual linear geometries also supported
	\item Parallel mesh reader with scalability for large meshes and processors
	\item Mesh partitioning using ParMetis \citeParMetis
	\item Unique physical tag for each element and domain boundary. Read from GMSH file and accessible via grid methods.
	\item Unique global index for entities of all codimensions.
	\item Ghost elements of all codimensions (optional)
	\item Communication protocols for all codimensions
\end{itemize}

\noindent
The following functionality is currently NOT supported. As seen below, some of this functionality will be implemented in the nearest future, some other is not currently foreseen. We welcome contributions from the community
\begin{itemize}
	\item $[1-2 months]$ Location of containing element by global coordinate (via OCTree)
	\item $[1/2 year]$  Does NOT support global and local refinement
	\item $[1/2 year]$  Does NOT support hanging nodes
	\item $[1/2 year]$  Does NOT support periodic boundaries at the moment
	\item $[1 year]$    Does Not support curvilinear meshes of non-uniform order
	\item $[Undefined]$ Does NOT support 1D and 2D geometries. 
	\item $[Undefined]$ Does NOT support non-tetrahedral meshes.
	\item $[Undefined]$ Does NOT support front/overlap elements at the moment
\end{itemize}

\subsection{Design decisions}
\label{section-outline-designdecisions}

\begin{itemize}
	\item User must provide globalId's for vertices and elements. [Automatically implemented by GMSH \citeGMSH]
	\item User must provide all boundary segments inside GMSH file.
\end{itemize}


\subsection{Internal Structure}
\label{section-outline-internalstructure}


Curvilinear Geometry:
\begin{itemize}
	\item Polynomial - Stores polynomials explicitly. Is able to perform basic arithmetic operations with polynomials, as well as differentiation and integration over a reference element.
	\item CurvilinearElementInterpolator - Interpolates over curvilinear entity using Lagrange polynomials. Given a set of interpolatory vertices, provides explicit polynomial and numerical functions to transform local to global coordinates.
	\item CurvilinearGeometryHelper - Incorporates a set of auxiliary functions, mostly helping to extract information about subentities of a curvilinear geometry.
	\item NumericalRecursiveInterpolationIntegrator - Internal adaptive integration scheme. Works by recursively subdividing the reference entity, and interpolating the integrand using 4th order polynomial.
	\item QuadratureIntegrator - Wrapper for the Dune quadrature integration scheme. Performs integration of a provided functor over a reference element. Iteratively increases the quadrature order, until the estimated integration error satisfies user-required.
	\item CurvilinearGeometry - Provides information about the curvilinear entity. Provides access to curvilinear vertices, local and global mappings, normals, subentity geometries and integration elements. One of the new features is providing explicit polynomial mappings, and analytical integration of polynomial integrands over entities.
\end{itemize}

Curvilinear Grid
\begin{itemize}
	\item CurvilinearGMSHReader - Reads a $.msh$ file in parallel and fills it into a CurvilinearGridFactory. The mesh is partitioned directly in the reader (using $ParMetis$) in order to avoid unnecessary communication. Thus each process only reads entities that belong on it. Additionally, it is possible to directly write the data to PVTU files after reading, using CurvilinearVTKWriter.
	\item CurvilinearVTKWriter - A tool allowing to write curvilinear entities in parallel to VTK, VTU and PVTU files. Each curvilinear entity is discretized using linear entities, which are then written. Each entity provides process rank, structural type, and physical tag for better visualisation.
	\item CurvilinearGridBase - A self-consistent grid, that is later wrapped around. This grid does not have any additional user classes. It stores information about entities of the grid and their interconnection, and provides functions to access this information via local and global entity indices. It also provides iterators to iterate over underlying containers.
		\subitem CurvilinearGridStorage - Auxiliary class that stores all of the information about the grid
		\subitem CurvilinearGridConstructor - Auxiliary class that constructs the grid, namely entities, their local and global indices, ghost elements and communication maps.
	\item CurvilinearGrid - Main grid class, according to $dune-grid$ standard. Provides additional access to curvilinear features of the entities, as well as to their physical tags.
	\item AllCommunicate - Auxiliary user-friendly class, providing $MPI_alltoallv$ communication protocol wrapper for templated POD. Also, a scalable version, namely nearest neighbour all-to-all communication wrapper is provided.
	\item VectorHelper - Auxiliary user-friendly class, performs set operations on sorted templated vectors (union, intersection, complement), as well as conversion to string.
\end{itemize}

Utilities
\begin{itemize}
	\item LoggingMessage - A tool used to write logging output for the grid
	\item LoggingTimer - A tool that times different stages of mesh reading and grid construction, and reports on statistics over all processes. It can be used to identify time-bottlenecks in grid construction, as well as load imbalances among processors.
\end{itemize}
