\subsection{Integration}
\label{sec:theory:integration}

In this section we discuss the computation of scalar and vector integrals. Any global integral can be subdivided into elementary integrals, and then parametrized using the local elementary coordinates
\[
	\int_{\Omega} f(\vec{x}) d^{\dim} x =
	\sum_e \int_e f(\vec{x}) d^{\dim} x =
	\sum_e \int_e f(\vec{r}_e) \mu_e(\vec{r}_e) d^{\dim} r_e \]
where $\vec{x}$ are the global coordinates, $\vec{r}_e$ are local coordinates of element $e$, $f(\vec{x})$ an arbitrary function defined over the domain, and the function $\mu_e(\vec{r}_e)$, associated with the change of coordinates, is called the integration element. In the remainder of the section we focus on elementary integrals, so we drop the element index $e$. \\

\noindent
In the original \dunegeom{} paradigm, the geometry class does not explicitly compute integrals, only the integration element$\mu(\vec{r})$. The user can compute the integral over the reference element by using a quadrature rule \cite{abramowitz+1970} provided in \dunegeom{}, or another external integration module. Any numerical integral, in particular the numerical quadrature, can written as a weighted sum
\[ \int f(\vec{r}) \mu(\vec{r}) d^{\dim} r = \sum_i f(\vec{r}_i) \mu(\vec{r}_i) w_i  \]
where the $r_i$ and $w_i$ are the quadrature (sampling) points and associated weights. The sampling points and weights are a property of the particular quadrature rule, and can be reused for all integrands over the reference element. Given a polynomial order $p$, one can construct a finite numerical quadrature, which will be exact for polynomial functions of order $p$ and below, and thus well approximate integral over any function, that is well-approximated by a polynomial. \\

\noindent
Numerical quadrature methods in practice are considerably faster than any other known method for geometry dimensions $\dim \leq 3$ \cite{schurer2003}, but they also have disadvantages. Many smooth functions, for example, sharply peaked, nearly singular, or those given by fractional polynomial order are known to have very slow Taylor series convergence, and hence may require very high quadrature order to achieve the desired accuracy. Also, numerical quadratures for non-trivial domains (e.g. simplices) have so far only been calculated to moderate orders. For example, Zhang et al \cite{zhang+2008} present order 21 triangular quadrature. Finding a numerical quadrature reduces to finding polynomial roots, which is known to be a hard problem in more than 1 dimension due to progressively higher decimal precision necessary to distinguish the roots from one another. One way to overcome these difficulties is to transform the integration domain to a cuboid using a \textit{Duffy} transform (for full derivation, see abstract \ref{section-abstract-duffy-transform})

\[ \int_0^1 \int_0^{1-x} f(x,y) dx dy = \int_0^1 \int_0^1 f(x, (1-x)t ) dx dt  \]

\noindent
The advantage of the cuboid geometry is that a quadrature rule can be constructed from a tensor product of 1D quadratures, which are readily available at relatively high orders. Quadrature rules created in this way have more points per order than the optimal rules, created specifically for the desired (e.g. simplex) geometries. At the time of writing, \dunegeom{} provides 1D quadratures up to order 61 and specific 2D and 3D simplex quadratures up to order 13. We are aware of the existence of advanced methods to improve performance of quadrature integration, such as sparse grids \cite{petras2000}, but they are beyond the scope of this paper. \\

%%%\subsection{Overview of available numerical methods}
%%%
%%%\noindent
%%%Below is presented a short summary of integration method types known to us: \\
%%%
%%%\noindent
%%%\textbf{Gaussian Quadrature}: The method available in DUNE.
%%%\begin{itemize}
%%%	\item This method calculates the integral as a linear product of the integrand $f(\vec{r})$ values at specific precomputed points $\vec{r}_i$ with specific precomputed weights $w_i$, namely $I = \sum_i w_i f(\vec{r}_i)$. Thus the main advantage of this method is its computational cost, which is small for low order polynomial integrands.
%%%	\item Optimal quadrature points are only available for small dimensions. Finding such point sets for high dimension polynomials is very involved and is known to suffer from finite precision of floating point arithmetic. Alternatively, a suboptimal point distribution can be obtained from a tensor product space of 1D point distributions, whose size grows exponentially with integration dimension.
%%%	\item Gaussian Quadrature is constructed with the idea of calculating exact integrals for integrands being polynomial up to a given order. However, when integrating over a curved boundary, thte integration elemen is a square root of a polynomial, and polynomials really badly approximate square root, especially for small arguments, which can easily happen for highly curved elements. Not to mention that one, in principle, could with to integrate arbitrary (within reason) functions over the element. Thus
%%%		\subitem - Can GQ estimate integration error for non-polynomial functions?
%%%		\subitem - Can it be made hierarchical to have control over error by refinement?
%%%		\subitem - What would be the convergence rate to compare with other methods?
%%%\end{itemize}
%%%
%%%\noindent
%%%\textbf{Interpolatory adaptive refinement}: The method currently implemented in LagrangeGeometry subclass
%%%\begin{itemize}
%%%	\item - Evaluates integral over element, approximating the integrand by an interpolatory polymomials of two hierarchical orders (2 and 4 at the moment) \textbf{[IMAGE HERE]}
%%%	\item - The running integration error is approximated by the difference between the analytical integrals calculated from these two interpolatory polynomials.
%%%	\item - The higher order element is split into into sub-elements of lower order, and the integration proceeds recursively.
%%%	\item - Every time an element is split, its previous running error is subtracted from the total error, and the running errors of the sub-elements are added to the total error. Thus, the integration is terminated when total approximated error is below selected tolerance. 	
%%%		\subitem - using heap structure ordered by the approximate error of the element. This way avoids recursion, and at every iteration selects the element which has worst error, then refines it.
%%%		\subitem - When splitting, the previously calculated points are not re-calculated but hierarchically re-used by sub-elements. The sub-element only needs to be refined to a higher hierarchical order, by adding more points.
%%%	\item \textit{Possible improvement - Performance}. As the the refined element does not check if the neighboring elements are also being refined, so they both sample on the boundary twice. Does there exist a method to store/find intersection refinements faster than just compute 2nd time. Using order 4 for every new refined triangle we sample 9 new points, out of which 2 are being wasted, thus $22\%$ inefficient.
%%%\end{itemize}
%%%
%%%\noindent
%%%\textbf{Monte-Carlo integration} - according to above mentioned paper, good for dimensions 7 and above.
%%%\begin{itemize}
%%%	\item Randomly samples function over element, integral is approximated by the average over the sample
%%%	\item natural error estimate using sample standard deviation
%%%	\item Stratified Sampling: if after a set number of iterations sample error is larger than expected, then function is highly non-uniform. Split element in equal parts and continue recursively.
%%%	\item Markov Chain Monte Carlo (MCMC): Uses random walk to sample the integrand, thus concentrating the sample points where the function varies most. Can use Metropolis-Hastings to also adapt the sampling distribution.
%%%\end{itemize}
%%%
%%%\noindent
%%%\textbf{Interpolatory Spline integration} - this is just another idea...
%%%\begin{itemize}
%%%	\item Makes grid over element, cubic-interpolates all consecutive partially-overlapping segments, integrates analytically over each segment.
%%%		\subitem -tricky part 1: when interpolatory segments intersect, with what weights to take the intersecting parts
%%%		\subitem -tricky part 2: how well does this method interpolate boundaries of the element (being close to edges and faces)
%%%		\subitem -tricky part 3: how to estimate error of integration and necessary grid step?
%%%		\subitem - Any way to do the refinement or error-control?
%%%\end{itemize}

\noindent
The integration functionality in \curvgeom{} addresses two additional problems:
\begin{itemize}
	\item Integrating polynomials of arbitrary order
	\item Integrating smooth functions with no a priori knowledge of the optimal polynomial approximation order.
\end{itemize}


\vspace{6pt}

\noindent
\textbf{Symbolic Integration} \\
%
\curvgeom{} implements a symbolic polynomial class, which is stored as a sum of monomials of a given order. Integrals over monomials of any given order can be computed analytically \cref{appendix-proof-simplexintegral}, and so can the integral over any arbitrary polynomial, which is a sum of monomial integrals. \\

\begin{table}[h]
\centering
\begin{tabular}{l | l}
\hline
Cuboid Integrals &
\begin{tabular}{@{}c@{}}
$ \int_0^1 x^i dx = \frac{1}{i+1} $ \\
$ \int_0^1 \int_0^1 x^i y^j dx dy = \frac{1}{(i+1)(j+1)} $ \\
$ \int_0^1 \int_0^1 \int_0^1 x^i y^j z^k dx dy dz = \frac{1}{(i+1)(j+1)(k+1)} $ \\
\end{tabular} \\ \hline
Simplex Integrals &
\begin{tabular}{@{}c@{}}
$ \int_0^1 x^i dx = \frac{1}{i+1} $ \\
$ \int_0^1 \int_0^{1-x} x^i y^j dx dy = \frac{i! j!}{(i + j + 2)!} $ \\
$ \int_0^1 \int_0^{1-x} \int_0^{1-x-y} x^i y^j z^k dx dy dz = \frac{i! j! k!}{(i + j + k + 3)!} $ \\
\end{tabular} \\
\end{tabular} \\
\captionsetup{width=0.8\textwidth}
\caption{Monomial integrals over cuboid and simplex reference elements. For derivation see \cref{appendix-proof-simplexintegral}}
\label{table:integration:monomialintegral}
\end{table}

\noindent
The \curvgeom{} polynomial class provides the exact integration functionality. \\




\noindent
\textbf{Adaptive Integration} \\
%
In its general form, a scalar integral over an element can be written as

\[\int f(\vec{x}) d^{\dim} x = \int f(\vec{r}) \mu(\vec{r}) d^{\dim} r,\]

\noindent
where the integration element is given by \[\mu(\vec{r}) = \sqrt{\det(J J^T)} \] and $J$ is the \textit{Jacobian} matrix (see \cref{appendix:integrationelements:proof}). \\

\noindent
In the case of matching element and space dimension, e.g. volume in 3D, or area in 2D, the integration element simplifies to $\mu(\vec{r}) = |\det J(\vec{r})|$. Even though the absolute value function is not polynomial, it can be observed that $\det J$ is not allowed to change sign inside the element, as that would result in self-intersection. The change of sign implies that the global geometry contains both "positive" and "negative" volume, which happens due to twisting the global coordinates inside out at the singular point $\det J = 0$. Also, the singular point $\det J = 0$ should not be present within the element, as it leads to zero volumes in global coordinates. Modern curvilinear meshing tools take account of these constraints when constructing meshes. Thus, in the case of well-conditioned elements, it remains to evaluate the integration element it anywhere inside the element and discard the minus sign if it happens to be negative. Then, given a polynomial integrand $f(\vec{x})$, the integral can be computed exactly using the quadrature rule of appropriate order. \\

\noindent
In the case of mismatching dimensions, e.g. area in 3D, or length in 2D and 3D, $\mu(\vec{r})$ cannot be simplified. It is a square root a polynomial that itself is not a square of another. Such integrals, in general, do not possess a closed form solution and have to be evaluated numerically. To address this problem, \curvgeom{} provides a recursive integrator class, which iteratively increases the quadrature order until the estimated integration error converges to a desired tolerance level. This method can consume several milliseconds for calculation of the surface area of near-singular geometries, but for well-conditioned geometries it converges much faster. The method accepts integrands in terms of functors overloading a skeleton class, and the \curvgeom{} uses it internally to provide volumes and surfaces of curvilinear entities, only requiring the user to additionally specify the desired tolerance level. \\

\noindent
In addition, the integrator class supports simultaneous integration of vector and matrix integrands via $Dune::DynamicVector$ and $Dune::DynamicMatrix$, as well as $std::vector$. The motivation of this implementation is due to the fact that, frequently, the simultaneous evaluation of a vector or a matrix is considerably cheaper than the individual evaluation of each of its components. The method provides several matrix and vector convergence error estimates, such as 1 and 2-norm, which can be selected by user to adapt to the problem at hand.\\

\noindent
According to \cite{schurer2003}, best results in low-dimensional numerical integration are achieved by the adaptive quadrature of high degree, whereas \textit{Monte-Carlo} methods perform better for high-dimensional integrals. Using an external adaptive library, for example the GSL extension due to Steven G. Johnson (\url{http://ab-initio.mit.edu/wiki/index.php/Cubature}) could be of advantage. This library is based on \textit{Clenshaw-Curtis} quadrature, which has the advantage of being hierarchical. This means that the computation of next quadrature order reuses all previously-computed quadrature points, decreasing the computational effort.

%%%\subsection{Integration Element - Vector}
%%%
%%%When integrating vector functions we are mostly interested in the integrals over boundary surfaces and edges, namely $\int_{\partial V} \vec{f}(\vec{r}) \cdot \vec{n}(\vec{r}) d(\partial V)$. For an edge in 2D the following expression for the tangential and normal integration elements (up to a sign convention) can be found:
%%%\[ d\vec{l}_{\parallel} = (\partial_u p_x, \partial_u p_y)du \; \; \; \; \; d\vec{l}_{\perp} = (\partial_u p_y, -\partial_u p_x)du  \]
%%%
%%%\noindent
%%%For a vector in 3D the tangential integration element is not defined, but the normal integration element is
%%%\[ d\vec{S} = (\partial_u \vec{p} \times \partial_v \vec{p})du \; dv  \]
%%%
%%%\noindent
%%%Thus, given polynomial vector basis functions $\vec{f}$ and polynomial interpolation, the scalar (and, if necessary, vector) products $\vec{f}(u) \cdot d\vec{l}(u)$ and $\vec{f}(u,v) \cdot d\vec{S}(u,v)$ are also polynomial, and can be integrated exactly using analytic polynomial integration code.
