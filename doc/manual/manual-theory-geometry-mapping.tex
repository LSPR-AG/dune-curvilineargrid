\subsection{Coordinate transformation}
\label{sec:theory:coordinatetransform}

In order to calculate the coordinate transformation properties, one requires the knowledge of the local-to-global map $\vec{p}(\vec{r})$ and its first partial derivatives. Currently, \curvgeom{} only provides \textit{Lagrange} polynomials themselves as hard-coded expressions. Their derivatives are not yet available as hard-coded quantities, and thus are constructed by differentiating the symbolic polynomial map. This is, naturally, a little slower than having hard-coded derivatives. The advantage of analytical formulation is that the user can further apply algebraic and differential operators to the symbolic map to obtain, for example, a \textit{Hessian} matrix of the transformation.  \\

\noindent
\textbf{Local-to-Global map}
%
Local-to-global map $\vec{p}(\vec{r})$ is computed numerically using hard-coded \textit{Lagrange} polynomials when the order is below or equal to 5, and through analytic procedures otherwise. \\

\noindent
\textbf{\textit{Jacobian} and Inverse \textit{Jacobian}}
%
The local-to-global mapping is represented by a vector of symbolic polynomials, further computing \textit{Jacobian} matrix $J_{ij}(\vec{r}_0) = \partial_{r_i} p_j (\vec{R}) |_{\vec{r}_0}$ using exact partial differentiation provided by the polynomial class. This results in a matrix of polynomials, which can be evaluated for the desired local coordinates. The \textit{Jacobian} inverse and the integration element are then computed numerically using the \dune{} provided linear algebra routines, the matrix inverse $J^{-1}$ and pseudo-determinant $dI = \sqrt{\det(JJ^T)}$ respectively (see \cref{appendix:integrationelements:proof}). \\

\noindent
\textbf{Global-to-Local map}
%
For polynomial elements, global-to-local map is the inverse of a polynomial map. Given the matching world and entity dimension, the method searches for the exact coordinate local to the element, that corresponds to the provided coordinate. Further, this method is extended to elements with $(dim_{elem} \leq dim_{world})$ by converting it to an optimization problem
\begin{equation}
  \label{eq-theory-mapping-optimization}
  \vec{r} : |\vec{p}(\vec{r}) - \vec{x} |^2 \rightarrow \min
\end{equation} 
searching for the local coordinate closest to the inverse of the desired global coordinate in terms of distance in global coordinates. \\

\noindent
While this problem is always uniquely solvable in linear case, in the curvilinear case it poses several additional challenges
\begin{itemize}
	\item The polynomial interpolatory map $\vec{p}(\vec{r})$ is strictly bijective inside the reference element, which must be ensured by the mesh generator. However, this need not be the case outside it. For example, $p(r) = r^2$ is a bijective 1D local-to-global map for an edge defined on $[0,1]$. However, the map is clearly non-invertible for all $p(r) \leq 0$, and thus asking for a local coordinate corresponding to the global coordinate $-1$ has no defined answer.
	\item Curvilinear geometries have singularities, e.g. $r = 0$ in the previous example. At these points the integration element is zero, which most simple iterative methods can not handle. It is expected that the meshing software provides curvilinear entities with non-singular geometries, since this would result in infinitesimal global volumes associated with finite local volumes, destabilizing optimization methods and integration routines. There is no restriction on the singularity being in the near vicinity of the element, which may be enough to significantly damage convergence.
	\item For $(dim_{elem} \leq dim_{world})$, the optimization problem given by \cref{eq-theory-mapping-optimization} is highly nonlinear. There may be multiple solutions, even uncountably many.
\end{itemize}

\noindent
For obvious reasons we will not solve the problem directly, as searching for the roots of a system of polynomial equations in 2D and 3D is well-known to be a challenging task \cite{canny+1989}. Instead, the problem is solved by a first order \textit{Gauss-Newton} method \cite{bjoerck+1996}, extending the implementation from \texttt{dune-multilineargeometry}. \\

\noindent
Based on an exchange with the \dune{} user community, we have realized that in order to satisfy all use cases we need to implement two distinct methods
\begin{itemize}
  \item Restrictive method, useful to those who want to find the element containing the global coordinate, as well as the local coordinate inside that element. If the provided global coordinate is inside the element, the method will return a success and a valid local coordinate. Otherwise, the method will return a fail and no coordinate at all, meaning that the global coordinate is not inside the element. This method also extends to lower dimension entities, finding the local coordinate within the element (!), which minimizes the distance to the provided global coordinate. Given a well-defined map (non-singular in the vicinity of the element), this method is guaranteed to converge.
  \item Non-restrictive method, useful to those who wish to extrapolate the global-to-local map beyond the reference element. This method searches for the inverse (or the distance minimizer) over the entire local domain. This is a best effort method - due to the above mentioned difficulties, it is expected to fail to converge for some maps and global coordinates. In this case, an exception is thrown.
\end{itemize}

\noindent
Below we outline the algorithm of the restrictive method:

\begin{mybox}
\begin{enumerate}
	\item Let $\vec{x}_0$ be the requested global coordinate
	\item Start with a local point $\vec{r}_0$ guaranteed to be inside the element (e.g. its center),
	\item Iteratively choose better approximations for local coordinate using \[\vec{r}_{n+1} = \vec{r}_n + \vec{d}(\vec{r}_n)\] where $\vec{d}(\vec{r}_n)$ is the least squares solution of
	        \[ J(\vec{r}_n) \vec{d}(\vec{r}_n) = \vec{p}(\vec{r}_n) \] and $J(\vec{r})$ is the \textit{Jacobian} matrix.
	\item The iterative process is finished when the global coordinate distance converges to a given tolerance level $\epsilon$ in terms of the two-norm
	        \[ \epsilon_n = |\vec{p}(\vec{r}_n) - \vec{x}_0 |^2 \leq \epsilon \]
	\item The iteration is terminated prematurely if there is enough evidence that the optimal vertex is outside the element. For this, two criteria are used: the running estimate being far outside the element \[|\vec{p}_0 - \vec{p}_i|_2 > 4 R_{elem}\] and the convergence speed being significantly slower than quadratic.
\end{enumerate}
\end{mybox}

\noindent
We are still looking to improve this method. It correctly predicts the global coordinates being inside and outside the element for most of our tests, but fails to identify the boundary points inside the element for certain cases.