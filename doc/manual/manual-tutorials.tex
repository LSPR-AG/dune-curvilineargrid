\section{Usage (Curvilinear Grid How-to)}
\label{usage-howto}

In order to learn the workings of curvilinear grid it is easiest to study the source code of relevant tutorials \index{tutorial} provided inside the curvilinear grid module.

\subsection{Tutorial 1 - Getting started}
\label{usage-howto-tutorial-gettingstarted}

In this tutorial we will create a Curvilinear Grid by reading it from a GMSH file. This and all other tutorials can be run both in serial and in parallel.
First we define the grid \\

\begin{mybox}
\begin{lstlisting}
  typedef Dune::CurvilinearGrid<dim, dimworld, ctype> GridType;
\end{lstlisting}
\end{mybox}

\noindent
where $dim=3$ and $dimworld=3$ are dimensions of the grid and the world containing the grid. Currently this is the only allowed setup. \\

\noindent
Afterwards, we construct the Curvilinear Grid Factory \\

\begin{mybox}
\begin{lstlisting}
  Dune::CurvilinearGridFactory<GridType> factory(withGhostElements, verbose, processVerbose, mpihelper);
\end{lstlisting}
\end{mybox}

\noindent
where $bool\ withGhostElements$ defines whether the Ghost Elements will be constructed. $bool\ verbose,\ processVerbose$ determine the master process and all other processes would write the debug output. \\

\noindent
Then the Parallel Curvilinear GMSH Reader is used to read the mesh into the factory. \\

\begin{mybox}
\begin{lstlisting}
    Dune::CurvilinearGmshReader< GridType >::read(factory, filename, mpihelper, verbose, processVerbose, writeReaderVTKFile, insertBoundarySegment); 
\end{lstlisting}
\end{mybox}

\noindent
where $filename$ is the name of the $.msh$ file. $bool\ writeReaderVTKFile$ option allows to write the mesh to parallel VTU files immediately after reading. $bool\ insertBoundarySegment$ enables inserting boundary segments from the GMSH file. Currently the switch must be true, and $.msh$ file must contain all boundary segments for the grid to work. \\

\noindent
Finally, the factory is used to create the grid and return a pointer to it \\
\begin{mybox}
\begin{lstlisting}
  factory.createGrid();
\end{lstlisting}
\end{mybox}


\subsection{Tutorial 2 - Traverse}
\label{usage-howto-tutorial-traverse}

This tutorial repeats the procedure from tutorial 1 to create the grid, after which it iterates over the grid and extracts relevant information from the curvilinear entities. Currently, there is no refinement, so only leaf iterators are available, which are defined the usual Dune way given a codimension $codim$ of the entities\\

\begin{mybox}
\begin{lstlisting}
  typedef typename LeafGridView::template Codim<codim>::Iterator EntityLeafIterator;
  EntityLeafIterator ibegin = leafView.template begin<codim>();
  EntityLeafIterator iend   = leafView.template end<codim>();
  
  for (EntityLeafIterator it = ibegin; it != iend; ++it) {...}
\end{lstlisting}
\end{mybox}


Now we would like to extract some relevant information from the iterator \\
\begin{mybox}
\begin{lstlisting}
  Dune::GeometryType gt              = it->type();
  LocalIndexType  localIndex         = grid.leafIndexSet().index(*it);
  GlobalIndexType globalIndex        = grid.template entityGlobalIndex<codim>(*it);
  PhysicalTagType physicalTag        = grid.template entityPhysicalTag<codim>(*it);
  InterpolatoryOrderType interpOrder = grid.template entityInterpolationOrder<codim>(*it);
\end{lstlisting}
\end{mybox}

The $GeometryType$ and $LocalIndex$ are standard in Dune. $GlobalIndex$ provides a unique integer for each entity of a given codimension, over all processes. $PhysicalTag$ is the tag associated with the entity as defined in GMSH. It can be used to relate to the material property of the entity, or to emphasize its belonging to a particular subdomain. Originally this information was only available for extraction through the reader directly. $InterpolatoryOrder$ is an integer denoting the polynomial interpolation order of the geometry of the entity. It is allowed to take values 1 to 5. \\






\subsection{Tutorial 3 - Visualisation}
\label{usage-howto-tutorial-visualisation}

Curvilinear VTK writer is a tool capable of writing curvilinear geometries to VTK, VTU and PVTU files. It has the following features
\begin{itemize}
	\item Works in serial and parallel
	\item Writes curvilinear edges, triangles and tetrahedra
	\item Curvilinear entities are discretized into sets of linear edges / triangles, which are then written
	\item Writes following parameters for each entity
	  \subitem - process rank    - the rank of containing process
	  \subitem - physical tag    - an integer associated with with each entity, for example its material property
	  \subitem - structural type - an integer which distinguishes between different partition types of the entity, such as Internal, Ghost, Domain and Process Boundaries
\end{itemize}

\noindent
Creation of the grid and extraction of its parameters is done in the same way as in tutorial 2. To write the VTK output, first the writer class needs to be initialized \\

\begin{mybox}
\begin{lstlisting}
  Dune::CurvilinearVTKWriter<GridType> vtkCurvWriter(verbose, processVerbose, mpihelper);
\end{lstlisting}
\end{mybox}

\noindent
Afterwards, the tags vector needs to be created by combining the parameters discussed above

\begin{mybox}
\begin{lstlisting}
  std::vector<int> tags  { physicalTag, structType, mpihelper.rank() };
\end{lstlisting}
\end{mybox}

\noindent
Now each entity of dimension $mydim$ is added to the writer using the command below. The corresponding parameters are described in the \textbf{[WRITER]} section. \\

\begin{mybox}
\begin{lstlisting}
  vtkCurvWriter.template addCurvilinearElement<mydim>(gt, interpVertices, tags, interpOrder, N_DISCRETIZATION_POINTS, interpolate, explode, WRITE_VTK_EDGES, WRITE_VTK_TRIANGLES);
\end{lstlisting}
\end{mybox}

\noindent
Finally, the data needs to be written to a file. For this one of the following 3 routines can be used \\

\begin{mybox}
\begin{lstlisting}
  vtkCurvWriter.writeVTK(filename);
  vtkCurvWriter.writeVTU(filename);
  vtkCurvWriter.writeParallelVTU(filename_without_extension);
\end{lstlisting}
\end{mybox}

\noindent
The first two would write the output into the specified VTK and VTU files correspondingly. The third one would write a VTU file and a PVTU file if on master process, and only VTU file if on any other process.




\subsection{Tutorial 4 - Recursive Numerical Integration}
\label{usage-howto-tutorial-integration-recursive}

\subsection{Tutorial 5 - Polynomial Manipulation and Integration}
\label{usage-howto-tutorial-polynomial}

\subsection{Tutorial 6 - Communication}
\label{usage-howto-tutorial-communication}


\subsection{Tutorial 7 - Point Location - OCTree}
\label{usage-howto-tutorial-octree}










