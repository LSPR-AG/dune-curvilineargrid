%%%%%%%%%%%%%%%%%%%%%%%%%%%%%%%%%%%%%%%%%%%%%%%%%%%%%%%%%%%%%%%%%%%
%%
%% objective - CSCS small project proposal
%%
%%%%%%%%%%%%%%%%%%%%%%%%%%%%%%%%%%%%%%%%%%%%%%%%%%%%%%%%%%%%%%%%%%%



%\documentclass[a4paper,twocolumn]{article}
%\documentclass[a4paper,10pt]{article}
\documentclass[12pt]{report}

\usepackage[sort&compress,super]{natbib}
\usepackage[english]{babel}
\usepackage{amsmath}
\usepackage{graphicx}
\usepackage{caption}
\usepackage{subcaption}
%\usepackage{subfigure}                                %% for creating nested figures within figures
\usepackage{graphicx}
\usepackage{url}
\usepackage{array}
\usepackage{algorithm,algorithmic}
\usepackage{tikz}
\usepackage{url}  
\usetikzlibrary{arrows}
\usetikzlibrary{mindmap,trees}
\usepackage[overlay,absolute]{textpos}
%\usepackage[latin1]{inputenc}
\usepackage{wrapfig}
\usepackage{makeidx}
\usepackage{listings}
\usepackage{color}
\usepackage{xcolor}
\usepackage{enumitem}
\usepackage[normalem]{ulem}
\usepackage{mathtools}
\usepackage{lscape}
\usepackage{hyperref}


\DeclarePairedDelimiter{\ceil}{\lceil}{\rceil}
\DeclarePairedDelimiter{\floor}{\lfloor}{\rfloor}


% Rounded boxes functionality
\usepackage[framemethod=tikz]{mdframed}
\definecolor{mycolor}{rgb}{0.122, 0.435, 0.698}

\newmdenv[innerlinewidth=0.5pt, roundcorner=4pt,linecolor=mycolor, backgroundcolor=lightgray, innerleftmargin=4pt,
innerrightmargin=4pt,innertopmargin=2pt,innerbottommargin=2pt]{mybox}



%\usepackage[colorlinks,linkcolor=blue,anchorcolor=blue,citecolor=green]{hyperref} % hyper reference to contents 

\citestyle{plain}

\makeindex



% The following parameters seem to provide a reasonable page setup.
\topmargin 0.0cm
\oddsidemargin 0.2cm
\textwidth 16.1cm 
\textheight 21cm
\footskip 1.0cm



\begin{document}

\lstset{language=C++, breaklines=true}



\begin{titlepage}




\begin{center}
    
\noindent \textsc{{\Large Dune-CurvilinearGrid}}

\vspace{5mm}

\noindent \textbf{\textsc{{\Large Parallel grid for unstructured tetrahedral curvilinear meshes}}}
  
\vspace{2mm}
    
{\large
    
\noindent Aleksejs Fomins$^{\mathrm{a,b}}$ and Benedikt Oswald$^{\mathrm{b}}$

  }

\vspace{1mm}

\noindent $^{\mathrm{a}}$ Nanophotonics and Metrology Laboratory (\texttt{nam.epfl.ch})
\noindent Ecole Polytechnique F\'ederale de Lausanne (EPFL)
  
\vspace{1mm}

\noindent $^{\mathrm{b}}$ LSPR AG, Technopark Z\"urich, Technoparkstrasse 1, CH-8005 Z\"urich
\noindent phone +41 43 366 90 74 - email: \texttt{aleksejs.fomins@lspr.ch} and \texttt{benedikt.oswald@lspr.ch}

\vspace{2mm}

\end{center}



%%%%%%%%%%%%%%%%%%%%%%%%%%%%%%%%%%%%%%%%%%%%%%%%%%%%%%%%%%%%%%%%%%%%%%%%
% ACKNOWLEDGEMENTS
%%%%%%%%%%%%%%%%%%%%%%%%%%%%%%%%%%%%%%%%%%%%%%%%%%%%%%%%%%%%%%%%%%%%%%%%



\vspace{15mm}
\noindent \textbf{Acknowledgements} - While the architecture, implementation and down-to-earth programming
work for \texttt{dune-curvilineargrid} grid manager is credited with Aleksejs Fomins and Benedikt Oswald,
both of them are pleased to acknowledge the inspiration and support from the wider
\text{DUNE} community. We mention names in alphabetical order and sometimes associated with a specific
subject. In case we have forgotten to acknowledge an important contributor we kindly ask you to inform us and we will be
happy to insert the name immediately. So, then, here we are:
%%
%%
\textit{Peter Bastian}, Professor at University of Heidelberg, Germany
initial suggestion to consider curvilinear tetrahedral grids in order to reduce the computational burden onto the complex linear solver;
%%
%%
\textit{Markus Blatt}, Heidelberg, Germany, based independent high performance computing and \text{DUNE} contractor,
numerous hints related to the build system and \text{DUNE} architecture;
%%
%%
\textit{Andreas Dedner}, professor, University of Warwick, United Kingdom,
numerous hints related to the \text{DUNE} architecture;
%%
%%
\textit{Jorrit 'Hippi Joe' Fahlke}, postdoctoral scientist, University of M\"unster, Germany,
numerous hints related to the \text{DUNE} architecture, grid API, grid testing
and many other fields;
%%
%%
\textit{Dominic Kempf}, Phd student, University of Heidelberg, Germany
support w.r.t grid API implementation in \text{DUNE};
%%
%%
\textit{Robert Kloefkorn}, senior research scientist, IRISI, Norway,
support w.r.t grid API implementation in \text{DUNE};
%%
%%
\textit{Martin Nolte}, postdoctoral scientist, University of Freiburg im Breisgau, Germany,
numerous hints related to the \text{DUNE} architecture;
%%
%%
\textit{Oliver Sander}, professor, TU Aachen, Germany,
numerous hints related to the \text{DUNE} architecture,
numerical integration and quadrature;





%%%%%%%%%%%%%%%%%%%%%%%%%%%%%%%%%%%%%%%%%%%%%%%%%%%%%%%%%%%%%%%%%%%%%%%%
% BY WHOM DEVELOPMENT WAS SPONSORED
%%%%%%%%%%%%%%%%%%%%%%%%%%%%%%%%%%%%%%%%%%%%%%%%%%%%%%%%%%%%%%%%%%%%%%%%



\vspace{20mm}
{\small
\noindent \textbf{LEGAL NOTICE} - The development of the \texttt{dune-curvilineargrid} grid manager is sponsored by \textbf{LSPR AG},
Technopark Z\"urich, Technoparkstrasse 1, CH-8005 Z\"urich. \textbf{LSPR AG} exclusively holds all rights associated with \texttt{dune-curvilineargrid}.
%%
%%
The \texttt{dune-curvilineargrid} fully parallel grid manager will be made publicly available via \texttt{Github}
and free software based on the GPLv2 license. Other licenses can be negotiated through
\texttt{benedikt.oswald@lspr.ch}.
%%
%%
We herewith exclude any liability on the part of \textbf{LSPR AG} since the software is made available as is. Any user uses the software
at his own risk and by no means \textbf{LSPR AG} assumes any responsibility for harmful consequences or any other damage caused
by the use of the software. In particular, we emphasise that the whole project is governed by Swiss Law and nothing else.  Especially,
we reject any attempt of any other sovereign law to cover what we do.
}









\pagebreak

%%%%%%%%%%%%%%%%%%%%%%%%%%%%%%%%%%%%%%%%%%%%%%%%%%%%%%%%%%%%%%%%%%%%%%%%
% BEGIN OF ABSTRACT
%%%%%%%%%%%%%%%%%%%%%%%%%%%%%%%%%%%%%%%%%%%%%%%%%%%%%%%%%%%%%%%%%%%%%%%%

\vfill

\noindent \textbf{\textsc{ABSTRACT}} - We introduce the \texttt{dune-curvilineargrid} module. The module provides the self-contained, parallel
grid manager \texttt{dune-curvilineargrid} for curvilinear tetrahedral elements up to $5^{\mathrm{th}}$  order and conforms to the \text{DUNE} grid API.
%%
In particular, \texttt{dune-curvilineargrid} is a grid manager in its own right which offers this functionality:
%%
%%
\textbf{(1)} fully parallel mesh input in the \texttt{Gmsh} mesh format, thus we achieve the scalable input of very large meshes;
i.e. when a large mesh is read from disk, every process in the parallel context reads a small piece of the mesh; thus, we avoid the
scalability bottleneck that would arise if the whole mesh were read on the master process and then communicated to other processes;
the latter approach would inevitably lead to memory congestion on the master node and crash the program;
%%
%%
\textbf{(2)} once the mesh has been read from disk, it is partitioned using the \texttt{Parmetis} graph partitioning tool;
%%
%%
\textbf{(3)} \texttt{dune-curvilineargrid} manages meshes that consist of tetrahedra with curvilinear faces and edges, up to
$5^{\mathrm{th}}$ order; the curvilinear mesh geometry is modelled through \textit{Lagrange} polynomials of the respective order;
%%
%%
\textbf{(4)} since \texttt{dune-curvilineargrid} is a fully parallel grid manager for distributed memory architectures, it provides
globally unique indices over all processes in the parallel context.
%%
%%
\textbf{(5)} the \texttt{dune-curvilineargrid} fully parallel grid manager provides ghost elements associated with the interprocessor
boundaries;
%%
%%
\textbf{(5)} the \texttt{dune-curvilineargrid} grid manager provides communication protocols via the standard
\text{DUNE} data handle interface for all codimensions; i.e. data associated with vertices, edges, faces and volumes can
be communicated.
%%
%%
%%
%%
The \texttt{dune-curvilineargrid} grid manager is continously developed and improved. Among other things, we are working
on local refinement, modelling hanging nodes, non-uniform curvilinear order, periodicity and internal boundaries which
may become interesting for the implementation of hybrid finite-element-boundary-integral (FEBI) discretisation schemes.





%%%%%%%%%%%%%%%%%%%%%%%%%%%%%%%%%%%%%%%%%%%%%%%%%%%%%%%%%%%%%%%%%%%%%%%%
% END OF ABSTRACT
%%%%%%%%%%%%%%%%%%%%%%%%%%%%%%%%%%%%%%%%%%%%%%%%%%%%%%%%%%%%%%%%%%%%%%%%





\end{titlepage}


\tableofcontents





\chapter{Introduction}

%%%%%%%%%%%%%%%%%%%%%%%%%%%%%%%%%%%%%%%%%%%%%%%%%%%%%%%%%%%%%%%%%%%%%
% Curvilinear Grid Outline - Section on capabilities of the Grid
%%%%%%%%%%%%%%%%%%%%%%%%%%%%%%%%%%%%%%%%%%%%%%%%%%%%%%%%%%%%%%%%%%%%%

\section{Outline}
\subsection{Capabilities}

\noindent
Currently the curvilinear grid supports the following functionality.
\begin{itemize}
	\item Self-consistent grid manager supporting 3D tetrahedral curvilinear grids.
	\item GMSH input files of curvilinear orders 1-5. Usual linear geometries also supported
	\item Parallel mesh reader with scalability for large meshes and processors
	\item Mesh partitioning using ParMetis \textbf{CITE}
	\item Unique physical tag for each element and domain boundary. Read from GMSH file and accessible via grid methods.
	\item Unique global index for entities of all codimensions.
	\item Ghost elements of all codimensions (optional)
	\item Communication protocols for all codimensions
\end{itemize}

\noindent
The following functionality is currently NOT supported. As seen below, some of this functionality will be implemented in the nearest future, some other is not currently foreseen. We welcome contributions from the community
\begin{itemize}
	\item $[1-2 months]$ Location of containing element by global coordinate (via OCTree)
	\item $[1/2 year]$  Does NOT support global and local refinement
	\item $[1/2 year]$  Does NOT support hanging nodes
	\item $[1/2 year]$  Does NOT support periodic boundaries at the moment
	\item $[1 year]$    Does Not support curvilinear meshes of non-uniform order
	\item $[Undefined]$ Does NOT support 1D and 2D geometries. 
	\item $[Undefined]$ Does NOT support non-tetrahedral meshes.
	\item $[Undefined]$ Does NOT support front/overlap elements at the moment
\end{itemize}

\subsection{Design decisions}

\begin{itemize}
	\item User must provide globalId's for vertices and elements. [Automatically implemented by GMSH]
	\item User must provide all boundary segments inside GMSH file.
\end{itemize}


\subsection{Internal Structure}


Curvilinear Geometry:
\begin{itemize}
	\item Polynomial
	\item CurvilinearElementInterpolator
	\item CurvilinearGeometryHelper
	\item NumericalRecursiveInterpolationIntegrator
	\item CurvilinearGeometry
\end{itemize}

Curvilinear Grid
\begin{itemize}
	\item CurvilinearGMSHReader
	\item CurvilinearVTKWriter
	\item CurvilinearGridBase
		\subitem CurvilinearGridStorage
		\subitem CurvilinearGridConstructor
	\item CurvilinearGrid
	\item AllCommunicate
	\item VectorHelper	
	
\end{itemize}









\chapter{Usage}

%%%%%%%%%%%%%%%%%%%%%%%%%%%%%%%%%%%%%%%%%%%%%%%%%%%%%%%%%%%%%%%%%%%%%
% Curvilinear Grid Howto - Section on tutorials
%%%%%%%%%%%%%%%%%%%%%%%%%%%%%%%%%%%%%%%%%%%%%%%%%%%%%%%%%%%%%%%%%%%%%

\section{Usage (Curvilinear Grid How-to)}
\label{usage-howto}

In order to learn the workings of curvilinear grid it is easiest to study the source code of relevant tutorials \index{tutorial} provided inside the curvilinear grid module.

\subsection{Tutorial 1 - Getting started}
\label{usage-howto-tutorial-gettingstarted}

In this tutorial we will create a Curvilinear Grid by reading it from a GMSH file. This and all other tutorials can be run both in serial and in parallel.
First we define the grid \\

\begin{mybox}
\begin{lstlisting}
  typedef Dune::CurvilinearGrid<dim, dimworld, ctype> GridType;
\end{lstlisting}
\end{mybox}

\noindent
where $dim=3$ and $dimworld=3$ are dimensions of the grid and the world containing the grid. Currently this is the only allowed setup. \\

\noindent
Afterwards, we construct the Curvilinear Grid Factory \\

\begin{mybox}
\begin{lstlisting}
  Dune::CurvilinearGridFactory<GridType> factory(withGhostElements, verbose, processVerbose, mpihelper);
\end{lstlisting}
\end{mybox}

\noindent
where $bool\ withGhostElements$ defines whether the Ghost Elements will be constructed. $bool\ verbose,\ processVerbose$ determine the master process and all other processes would write the debug output. \\

\noindent
Then the Parallel Curvilinear GMSH Reader is used to read the mesh into the factory. \\

\begin{mybox}
\begin{lstlisting}
    Dune::CurvilinearGmshReader< GridType >::read(factory, filename, mpihelper, verbose, processVerbose, writeReaderVTKFile, insertBoundarySegment); 
\end{lstlisting}
\end{mybox}

\noindent
where $filename$ is the name of the $.msh$ file. $bool\ writeReaderVTKFile$ option allows to write the mesh to parallel VTU files immediately after reading. $bool\ insertBoundarySegment$ enables inserting boundary segments from the GMSH file. Currently the switch must be true, and $.msh$ file must contain all boundary segments for the grid to work. \\

\noindent
Finally, the factory is used to create the grid and return a pointer to it \\
\begin{mybox}
\begin{lstlisting}
  factory.createGrid();
\end{lstlisting}
\end{mybox}


\subsection{Tutorial 2 - Traverse}
\label{usage-howto-tutorial-traverse}

This tutorial repeats the procedure from tutorial 1 to create the grid, after which it iterates over the grid and extracts relevant information from the curvilinear entities. Currently, there is no refinement, so only leaf iterators are available, which are defined the usual Dune way given a codimension $codim$ of the entities\\

\begin{mybox}
\begin{lstlisting}
  typedef typename LeafGridView::template Codim<codim>::Iterator EntityLeafIterator;
  EntityLeafIterator ibegin = leafView.template begin<codim>();
  EntityLeafIterator iend   = leafView.template end<codim>();
  
  for (EntityLeafIterator it = ibegin; it != iend; ++it) {...}
\end{lstlisting}
\end{mybox}


Now we would like to extract some relevant information from the iterator \\
\begin{mybox}
\begin{lstlisting}
  Dune::GeometryType gt              = it->type();
  LocalIndexType  localIndex         = grid.leafIndexSet().index(*it);
  GlobalIndexType globalIndex        = grid.template entityGlobalIndex<codim>(*it);
  PhysicalTagType physicalTag        = grid.template entityPhysicalTag<codim>(*it);
  InterpolatoryOrderType interpOrder = grid.template entityInterpolationOrder<codim>(*it);
\end{lstlisting}
\end{mybox}

The $GeometryType$ and $LocalIndex$ are standard in Dune. $GlobalIndex$ provides a unique integer for each entity of a given codimension, over all processes. $PhysicalTag$ is the tag associated with the entity as defined in GMSH. It can be used to relate to the material property of the entity, or to emphasize its belonging to a particular subdomain. Originally this information was only available for extraction through the reader directly. $InterpolatoryOrder$ is an integer denoting the polynomial interpolation order of the geometry of the entity. It is allowed to take values 1 to 5. \\






\subsection{Tutorial 3 - Visualisation}
\label{usage-howto-tutorial-visualisation}

Curvilinear VTK writer is a tool capable of writing curvilinear geometries to VTK, VTU and PVTU files. It has the following features
\begin{itemize}
	\item Works in serial and parallel
	\item Writes curvilinear edges, triangles and tetrahedra
	\item Curvilinear entities are discretized into sets of linear edges / triangles, which are then written
	\item Writes following parameters for each entity
	  \subitem - process rank    - the rank of containing process
	  \subitem - physical tag    - an integer associated with with each entity, for example its material property
	  \subitem - structural type - an integer which distinguishes between different partition types of the entity, such as Internal, Ghost, Domain and Process Boundaries
\end{itemize}

\noindent
Creation of the grid and extraction of its parameters is done in the same way as in tutorial 2. To write the VTK output, first the writer class needs to be initialized \\

\begin{mybox}
\begin{lstlisting}
  Dune::CurvilinearVTKWriter<GridType> vtkCurvWriter(verbose, processVerbose, mpihelper);
\end{lstlisting}
\end{mybox}

\noindent
Afterwards, the tags vector needs to be created by combining the parameters discussed above

\begin{mybox}
\begin{lstlisting}
  std::vector<int> tags  { physicalTag, structType, mpihelper.rank() };
\end{lstlisting}
\end{mybox}

\noindent
Now each entity of dimension $mydim$ is added to the writer using the command below. The corresponding parameters are described in the \textbf{[WRITER]} section. \\

\begin{mybox}
\begin{lstlisting}
  vtkCurvWriter.template addCurvilinearElement<mydim>(gt, interpVertices, tags, interpOrder, N_DISCRETIZATION_POINTS, interpolate, explode, WRITE_VTK_EDGES, WRITE_VTK_TRIANGLES);
\end{lstlisting}
\end{mybox}

\noindent
Finally, the data needs to be written to a file. For this one of the following 3 routines can be used \\

\begin{mybox}
\begin{lstlisting}
  vtkCurvWriter.writeVTK(filename);
  vtkCurvWriter.writeVTU(filename);
  vtkCurvWriter.writeParallelVTU(filename_without_extension);
\end{lstlisting}
\end{mybox}

\noindent
The first two would write the output into the specified VTK and VTU files correspondingly. The third one would write a VTU file and a PVTU file if on master process, and only VTU file if on any other process.




\subsection{Tutorial 4 - Quadrature Integration Tutorial}
\label{usage-howto-tutorial-integration-quadrature}

Here we demonstrate a way to integrate a function over the boundary of the domain. In this example we are going to check Gauss law by computing the surface integral of the electric field of a unit point charge across the domain boundary of the mesh enclosing that charge. The tutorial will demonstrate that changing the curvature of the domain boundary does not affect the result. \\

\noindent
We would like to compute the integral
\[\int_{\delta \Omega} \vec{E}(\vec{x}) \cdot d\vec{S} = \int_{\delta \Omega} \vec{E}(\vec{x}(\vec{u})) \cdot \vec{n}(\vec{u}) \Delta J(\vec{u}) d^2 u \]
\noindent
where $\vec{x}$ is the global coordinate, $\vec{u}$ is the coordinate local to the surface finite element, $\vec{E}(\vec{x}) = |\vec{x}|^{-2} \vec{e}_r$ is the electric field of a unit charge located at the origin of global coordinates, $\vec{n}(\vec{u})$ is the surface outer normal in global coordinates as function of local coordinate and $\Delta J(\vec{u}) = \sqrt{\det [ J^T(\vec{u}) J(\vec{u}) ]}$ is the generalized integration element which comes from conversion of the integral from global to local coordinates. \\

\noindent
Below we present a functor which calculates the integrand for the above integral for a given boundary face. We initialize it with the intersection class, as it contains all the necessary information to perform the calculation. The function $ChargeField(x)$ calculates $\vec{E}(\vec{x})$. The important is the $operator()(x)$, which calculates the integrand. Given a local coordinate, it first finds integration outer normal at that point, which is equivalent to $\vec{n}(\vec{u}) \Delta J(\vec{u})$. Then it calculates the associated global coordinate and the field at that global coordinate. Finally, it returns the scalar product between the integration normal and the field \\


\begin{mybox}
\begin{lstlisting}
  template<class Grid, int mydim>
  struct GaussFunctor
  {
    typedef Dune::FieldVector<double, mydim>  LocalCoordinate;
    typedef typename Grid::Traits::LeafIntersection               Intersection;

    Intersection I_;

    GaussFunctor(const Intersection & I) : I_(I)  {}

    static GlobalCoordinate ChargeField(const GlobalCoordinate & x)
    {
      GlobalCoordinate rez = x;
      rez /= pow(x.two_norm2(), 1.5);
      return rez;
    }

    double operator()(const LocalCoordinate & x) const
    {
      GlobalCoordinate integrnormal = I_.integrationOuterNormal(x);
      GlobalCoordinate global = I_.geometry().global(x);
      GlobalCoordinate field = ChargeField(global);

      double rez = 0;
      for (int i = 0; i < 3; i++) { rez += integrnormal[i] * field[i]; }
      return rez;
    }
  };
\end{lstlisting}
\end{mybox}

\noindent
Now, in order to calculate the Gauss law integral, the above functor needs to be calculated for all domain boundary segments and integrated over the reference element (reference triangle for simplex meshes). Thus, the code iterates over all elements of the mesh, then over all intersections, then constructs the above functor using each intersection that does not have a neighbour, and integrates it over the reference element using the gradually approximating quadrature as given in \ref{interface-integrator-quadrature}. \\

\begin{mybox}
\begin{lstlisting}
  double gaussintegral = 0.0;
  double rel_tol = 1.0e-5;

  LeafGridView leafView = grid.leafGridView();
  EntityLeafIterator ibegin = leafView.template begin<0>();
  EntityLeafIterator iend   = leafView.template end<0>();

  for (EntityLeafIterator it = ibegin; it != iend; ++it)
  {
    const Entity &entity = *it;
    const IntersectionIterator nend = leafView.iend(entity);
    for( IntersectionIterator nit = leafView.ibegin(entity); nit != nend; ++nit )
    {
      const Intersection &intersection = *nit;

      if (!intersection.neighbor())
      {
        Dune::GeometryType gt = intersection.type();
        GaussFunctor<GridType, 2> f(intersection);
        Dune::QuadratureIntegrator<double, 2> qInt;
        double thisIntegral = qInt.integrateRecursive(gt, f, rel_tol).second;
        gaussintegral += thisIntegral;
      }
    }
  }
\end{lstlisting}
\end{mybox}





%\subsection{Tutorial 4 - Recursive Numerical Integration}
%\label{usage-howto-tutorial-integration-recursive}

\subsection{Tutorial 5 - Polynomial Manipulation and Integration}
\label{usage-howto-tutorial-polynomial}

\subsection{Tutorial 6 - Communication}
\label{usage-howto-tutorial-communication}


\subsection{Tutorial 7 - Point Location - OCTree}
\label{usage-howto-tutorial-octree}













%%%%%%%%%%%%%%%%%%%%%%%%%%%%%%%%%%%%%%%%%%%%%%%%%%%%%%%%%%%%%%%%%%%%%
% Curvilinear Grid Diagnostics - Section on mesh statistics and visualisation
%%%%%%%%%%%%%%%%%%%%%%%%%%%%%%%%%%%%%%%%%%%%%%%%%%%%%%%%%%%%%%%%%%%%%

\section{Diagnostics tools}
\label{section-diagnostics}


\subsection{Mesh statistics}
\label{section-diagnostics-statistics}


\subsection{Visualisation}
\label{section-diagnostics-visualisation}









\chapter{Interface - Curvilinear Geometry}

%%%%%%%%%%%%%%%%%%%%%%%%%%%%%%%%%%%%%%%
% Implementation of Polynomial Class
%%%%%%%%%%%%%%%%%%%%%%%%%%%%%%%%%%%%%%%
\section{Polynomial Class}

\noindent
Arbitrary polynomial of order up to $n$ with $d$ parameters can be represented in its expanded form as
\[ p(\vec{u}) = \sum_i A_i \prod_{j = 0}^d u_j^{\mathrm{pow}_{i,j}},  \]
for example in 3D this can be written as
\[ p(\vec{u}) = \sum_i A_i u^{pow_{u,i}} v^{pow_{v,i}} w^{pow_{w, i}},  \]

\noindent
Therefore, we define a PolySummand class, which stores a constant multiplier $A$ and powers vector $pow$, and Polynomial class
which stores a vector of PolySummands, and does required operations on them. Below we describe the implemented functionality

\subsection{Methods}

\begin{itemize}
	\item \uline{Initialization}: empty or with a single summand
	\item \uline{Operators}: adding, subtracting or multiplying 2 polynomials or polynomial and a scalar.
		\subitem - to multiply by constant, need to loop and multiply each $A_i$ by constant
		\subitem - to add or subtract 2 polynomials, merge the summand vectors and compactify
		\subitem - to multiply 2 polynomials, constructing a new polynomial from pairwise products of all summands, and compactify
	\item \uline{Evaluate}: evaluates all summand and adds them up.
	\item \uline{Take derivative}: lowers the corresponding power for each summand, multiplies prefactor by that power, and removes the summands that differentiate to 0
	\item \uline{Compactify}: adds up all summands with the same power. Sorts the summands by $(x_1,y_1,z_1) < (x_2, y_2, z_2)$, where $x$ has the highest priority and $z$ has the lowest priority. Then all of the repeating powers will be consecutive. Simply loop over sorted polynomial, and to a new polynomial add the sums of all consecutive repeating polynomials.
	\item \uline{Integrate}: Integration of a polynomial over a reference simplex. Naturally, it is the sum of integrals of summands. It can be shown that
		\subitem - 1D integral over the reference edge is $A \frac{1}{pow_u + 1}$
		\subitem - 2D integral over the reference triangle is $A \frac{pow_u! pow_v!}{(pow_u + pow_v + 2)!}$
		\subitem - 3D integral over the reference tetrahedron is $A \frac{pow_u! pow_v! pow_w!}{(pow_u + pow_v + pow_w + 3)!}$
\end{itemize}

\subsection{Tests}

\noindent
Currently the tests are only for 1, 2 and 3 dimensions. Most of the tests use intrinsic functionality like polynomial operators and derivatives to construct polynomials and print them to the screen, and request the user to to verify manually if they match the expected polynomials which are also printed. For each dimension there is one test which integrates a non-linear polynomial over simplex and prints out the result which is also compared manually. \\

\textbf{TODO:} These tests can and should be automatized in the future using integer string comparison. The test program should throw an error if a test fails


%%%%%%%%%%%%%%%%%%%%%%%%%%%%%%%%%%%%%%%
% Implementation of Curvilinear Geometry Helper
%%%%%%%%%%%%%%%%%%%%%%%%%%%%%%%%%%%%%%%
\include{manual-geometry-helper}


%%%%%%%%%%%%%%%%%%%%%%%%%%%%%%%%%%%%%%%
% Implementation of Interpolator Class
%%%%%%%%%%%%%%%%%%%%%%%%%%%%%%%%%%%%%%%
\section{Polynomial Interpolator Class}

\subsection{Methods}

\begin{itemize}
	\item \uline{Initialization:} Requires reference element/GeometryType, interpolatory vertex vector and interpolation order
	\item \uline{dofPerOrder:} Number of interpolatory vertices required for a given element type, dimension and interpolation order. For a simplex this quantity is
	\[ {DoF}_{dim}^{ord} = \{ \{ 2,3,4,5,6 \}, \{ 3,6,10,15,21 \}, \{ 4,10,20,35,56 \} \} \]
	\item \uline{cornerID:} The id number of a corner in the interpolatory vertex vector. For the simplices is calculated as follows:
		\subitem -for edge $\{ 0, \; ord \}$
		\subitem -for triangle $\{0, \; ord, \; {DoF}_{dim}^{ord} - 1\}$
		\subitem -for tetrahedron $\{0, \; ord, \; ord (ord + 3) / 2, \; {DoF}_{dim}^{ord} - 1\}$
	\item \uline{simplexGrid:} These 3 methods implement the functionality discussed in section \ref{subsection-simplexgrid}. They return a vector of indices which the grid points take in the d-dimensional matrix, and that vector divided by the interpolation order, which is exactly the local coordinates of the reference simplex grid.
	\item \uline{lagrangePolynomial:} Evaluates the i-th Lagrange Polynomial $L_i(\vec{r})$ for a given $i$ and a given local coordinate. Lagrange polynomials in this case are given explicitly for all orders to accellerate computation.
	\item \uline{realCoordinate:} Evaluates the global coordinate given a local coordinate. Computes the scalar product $\vec{p}(\vec{r}) = \sum_i \vec{p}_i L_i (\vec{r})$.
	\item \uline{interpolatoryVectorAnalytical:} Produces an analytical map from local to global coordinates in terms of polynomial vector.
		\subitem 0) Constructs local grid points $\vec{r}_i$ as given in \ref{subsection-simplexgrid}.		
		\subitem 1) Constructs monomial basis $\vec{z}^{ord}(\vec{r})$ as given in the beginning of section \ref{section-interppoly}
		\subitem 2) Evaluates all monomials $\vec{z}^{ord}(\vec{r})$ at all local grid points $\vec{r}_i$, assembling the DynamicMatrix V
		\subitem 3) Computes all lagrange polynomials using $\vec{L}(\vec{r}) = V^{-1} \vec{z}^{ord}(\vec{r})$
		\subitem 4) Computes the analytical map $\vec{p}(\vec{r}) = \sum_i \vec{p}_i L_i (\vec{r})$.
	\item \uline{SubentityInterpolators} Constructs Interpolator classes for each $\dim - 1$ subentity of the given element. Only allowed for elements of dimensions 2 and 3.
		\subitem - At the moment a rather crude algorithm is employed. To find the vertices corresponding to a given boundary, the order numbers of simplexGrid are written in a $(ord + 1) \times (ord + 1) \times (ord + 1)$ matrix, which is then used to easier locate the indices of the boundary interpolatory points in the vertex vector.
		\subitem - Certain orientation convention is chosen for the provided sub-entitiy interpolators, to simplify the future calculation of the outwards normals. The convention for edge orientations for a triangle $(012)$ are $(01)$, $(12)$ and $(20)$. The convention for triangle orientations for a tetrahedron $(0123)$ are $(012)$, $(023)$, $(213)$ and $(031)$.
\end{itemize}

\begin{figure}[hp]
    \centering
    \includegraphics[scale=0.5]{doc-pics/pic-subentity-interpolators-method.png}
    %\caption{Awesome Image}
    %\label{fig:awesome_image}
\end{figure}


\subsection{Tests}

\noindent
For each dimension several linear and polynomial Functors are defined which act as pre-defined local-to-global maps. Then, for a simplex of each dimension there is a testing routine which is run for every applicable Functor. The tests are as follows:
\begin{itemize}
	\item For each order generate a local grid and sample the given functor to obtain global coordinates for interpolation points and thus construct the interpolator.
	\item First test requests a global coordinate for each local grid point, both using explicit function $realCoordinate$ and by evaluating the analytical polynomial provided by $interpolatoryVectorAnalytical$. Then the 3 results are compared. For this test, all 3 results must match independent of Functor and interpolation order.
	\item Second test requests a global coordinate for a random set of local coordinates, also comparing the correct result with explicit and analytical functionality. Explicit and analytical results should be equal to each other for any test since they do the same thing. However, they will match to the true result only if the polynomial order of the Functor is lower or equal to the one being tested, and most likely should fail for lower orders.
\end{itemize}

\textbf{TODO:}
\begin{itemize}
	\item The tests must be automatized such that the program throws an error if a test fails.
	\item Would be useful to test the method $SubentityInterpolators$ which is not tested at the moment, but is indirectly tested later in the LagrangeGeometry tests.
\end{itemize}


%%%%%%%%%%%%%%%%%%%%%%%%%%%%%%%%%%%%%%%
% Implementation of Recursive Integration
%%%%%%%%%%%%%%%%%%%%%%%%%%%%%%%%%%%%%%%
\section{Recursive Integrator Class}

Evaluates integral over element, approximating the integrand by an interpolatory polymomials of two hierarchical orders (2 and 4 at the moment). The running integration error is approximated by the difference between the analytical integrals calculated from these two interpolatory polynomials. The higher order element is split into into sub-elements of lower order, and the integration proceeds recursively. Every time an element is split, its previous running error is subtracted from the total error, and the running errors of the sub-elements are added to the total error. Thus, the integration is terminated when total approximated error is below selected tolerance. Heap structure ordered by the approximate error of the element is used to avoid recursion. At every iteration the element with the worst error is selected and then refined. When splitting, the previously calculated points are not re-calculated but hierarchically re-used by sub-elements. The sub-element only needs to be refined to a higher hierarchical order, by adding more points. \\

\noindent
\textit{Possible improvement - Performance}. As the the refined element does not check if the neighboring elements are also being refined, so they both sample on the boundary twice. Does there exist a method to store/find intersection refinements faster than just compute 2nd time. Using order 4 for every new refined triangle we sample 9 new points, out of which 2 are being wasted, thus $22\%$ inefficient.

\begin{figure}[p]
    \centering
    \includegraphics[scale=0.8]{doc-pics/pic-numerical-integration-adaptive-interpolation.png}
    %\caption{Awesome Image}
    %\label{fig:awesome_image}
\end{figure}


\subsection{Methods}

\noindent
Numerical Integration is only available for Simplices at the moment.

\noindent
\begin{itemize}
	\item \uline{Initialization:} Requires GeometryType to know the type of the element being integrated
	\item \uline{Integrate:} Integrates a function given by a functor object, until the expected integration error is below given tolerance.
\end{itemize}


\subsection{Tests}

\noindent
Numerical Integration is only implemented and hence tested for edges and triangles. A multitude of integrands are given in terms of Functors, starting with simple polynomial integrands and ending with integrands involving square roots of polynomials, as this is these are functions the method is expected to integrate well. The integrals are computed and compared to expected ones, which are sometimes given explicitly in numerical form, except of a lucky integral of $\iint \sqrt{xy} \; dxdy$ which happens to be $\pi / 24$. \\

\textbf{TODO:}
\begin{itemize}
	\item These tests should be automatized, which is easy to do, as they are only comparing numerical values.
	\item Integrals of complicated functions may exceed 100000 samples for this method given relative precision $10^{-5}$, which is too slow for the desired application
\end{itemize}


%%%%%%%%%%%%%%%%%%%%%%%%%%%%%%%%%%%%%%%
% Implementation of Recursive Integration
%%%%%%%%%%%%%%%%%%%%%%%%%%%%%%%%%%%%%%%
\section{Quadrature Integrator Class}
\label{interface-integrator-quadrature}

The Quadrature Integrator class provides static members to integrate functors over simplex reference elements by wrapping the quadrature rules provided by Dune. At the moment the dune-default Gauss-Legendre quadrature is used, however, the interface can be easily extended with flexibility to select the desired quadrature if need arises. The integrand functor must provide the operator \\

\begin{mybox}
\begin{lstlisting}
  double operator()(const Dune::FieldVector<ctype, mydim> & x)
\end{lstlisting}
\end{mybox}

\noindent
where $mydim$ must be equal to $gt.dim()$. Below 3 functions provide different strategies to approach integration \\

\begin{mybox}
\begin{lstlisting}
  template<class Functor>
  static ctype integrate(Dune::GeometryType gt, Functor f, int integrOrder)
  template<class Functor>
  static StatInfoVec integrateStat(Dune::GeometryType gt, Functor f, int integrOrderMax)
  template<class Functor>
  static StatInfo integrateRecursive(Dune::GeometryType gt, Functor f, ctype rel_tol)
\end{lstlisting}
\end{mybox}

\noindent
The first method integrates functor over the reference element using quadrature of a given order. The second method repeats the first method for all quadrature orders from 1 to specified maximal order, and outputs a vector of pairs (order, result). The last method integrates the functor with gradually increasing order until the desired estimated error tolerance is reached, or the maximal allowed quadrature order is reached. The estimated relative order is calculated as $\bigl | 1 - \frac{I(o)}{I_{smooth}(o-1)}  \bigr |$, where the smooth integral is defined as $I_{smooth}(o) = \zeta (I(o+1) + I(o-1)) + (1 - 2\zeta)I(o)$, with $\zeta$ being the smoothing parameter with default value $\zeta = 0.15$. The basic idea is to estimate the error as the difference between the integrals at consecutive orders.
\begin{itemize}
	\item Improvement 1: For some consecutive orders in Dune the number of quadrature points does not change, in this case simply skip to the next order.
	\item Improvement 2: Sometimes by random chance the two consecutive orders give very close results, while the integral still has not converged. To partially avoid this scenario, the smoothing parameter is introduced. The idea is to represent the previous best guess as a weighted average between a few previous estimates. So, if the integral has made a sharp jump recently, we will doubt that the integral has converged, thus the expected error will be larger.
\end{itemize}

\textbf{[TODO]}: To accelerate the gradual approximation strategy, it would be useful to implement a hierarchic quadrature. Such quadrature would double the interpolation order at every step, and reuse all the interpolation points already calculated.



%%%%%%%%%%%%%%%%%%%%%%%%%%%%%%%%%%%%%%%
% Implementation of Curvilinear Geometry
%%%%%%%%%%%%%%%%%%%%%%%%%%%%%%%%%%%%%%%
\section{Curvilinear Geometry Base Class}
\label{interface-curvilineargeometry}

Curvilinear Geometry can be initialized either using the interpolator class, or the parameters necessary to initialize the interpolator class \\

\begin{mybox}
\begin{lstlisting}
  CurvilinearGeometry ( const ElementInterpolator & elemInterp)
  CurvilinearGeometry ( const ReferenceElement &refElement, const Vertices &vertices, InterpolatoryOrderType order)
  CurvilinearGeometry ( Dune::GeometryType gt, const Vertices &vertices, InterpolatoryOrderType order)
\end{lstlisting}
\end{mybox}


Standard dune functionality \\
\begin{mybox}
\begin{lstlisting}
  bool affine ()
  GlobalCoordinate center ()
\end{lstlisting}
\end{mybox}


Wrapper for basic interpolator functionality \\
\begin{mybox}
\begin{lstlisting}
  InterpolatoryOrderType order()
  Dune::GeometryType type()
  int nCorner ()
  int nVertex ()
  GlobalCoordinate corner ( InternalIndexType cornerLinearIndex )
  GlobalCoordinate vertex (int i)
  std::vector< GlobalCoordinate > cornerSet()
  std::vector<GlobalCoordinate> vertexSet()
\end{lstlisting}
\end{mybox}


Wrapper for extended interpolator functionality \\
\begin{mybox}
\begin{lstlisting}
  ElementInterpolator interpolator()
  PolynomialVector interpolatoryVectorAnalytical()

  template<int subdim>
  CurvilinearGeometry< ctype, subdim, cdim>  subentityGeometry(InternalIndexType subentityIndex)
\end{lstlisting}
\end{mybox}


Mappings \\
\begin{mybox}
\begin{lstlisting}
  GlobalCoordinate global ( const LocalCoordinate &local )
  bool local ( const GlobalCoordinate &globalC, LocalCoordinate & localC )
\end{lstlisting}
\end{mybox}


Construction of normals \\
\begin{mybox}
\begin{lstlisting}
  GlobalCoordinate normal(const LocalCoordinate &local )
  GlobalCoordinate subentityNormal(InternalIndexType indexInInside, const LocalCoordinate &local )
  GlobalCoordinate subentityUnitNormal(InternalIndexType indexInInside, const LocalCoordinate &local )
  GlobalCoordinate subentityIntegrationNormal(InternalIndexType indexInInside, const LocalCoordinate &local )
\end{lstlisting}
\end{mybox}


Construction numerical and analytic integration elements. \\

\noindent
\uline{JacobianDeterminantAnalytical}: $|\det J_{ij}|$ is available explicitly in polynomial form, when $(dim_{elem} = dim_{world})$. Although modulus is not a polynomial operation, $\det J_{ij} \neq 0$ inside the element, because the geometry must not be self-intersecting. Hence $\det J_{ij}$ is not allowed to change sign within the element, and modulus-correction is to evaluate $\det J_{ij}$ anywhere inside the element and multiply the analytic expression by $-1$ if the result is negative. \\

\noindent
\uline{NormalIntegrationElementAnalytical}: Surface normal integration element $d\vec{S} = \vec{n} dS $ is available. For edges in 2D defined by $d\vec{l} = (\partial_u p_y, -\partial_u p_x) du$. For triangles in 3D defined by $d \vec{S} = -(\partial_u \vec{p} \times \partial_v \vec{p}) du dv $ \\

\noindent
\uline{IntegrationElementSquaredAnalytical}: The square of the pseudodeterminant $\det(JJ^T)$, when $(dim_{elem} \neq dim_{world})$. For edges this expression is equal to $|\partial_u \vec{p}|^2$. For triangles in 3D this expression is equal to $|\vec{dS} / (du \; dv)|^2$. \\

\begin{mybox}
\begin{lstlisting}
  ctype integrationElement ( const LocalCoordinate &local )
  JacobianTransposed jacobianTransposed ( const LocalCoordinate &local )
  JacobianInverseTransposed jacobianInverseTransposed ( const LocalCoordinate &local )
  LocalPolynomial JacobianDeterminantAnalytical()
  PolynomialVector NormalIntegrationElementAnalytical()
  LocalPolynomial IntegrationElementSquaredAnalytical()
\end{lstlisting}
\end{mybox}


Numerical and analytic integration. $volume()$ is defined by integrating $f(\vec{r}) = 1$ over a scalar integration element. When dimension of element is equal to the world dimension, the integration is done analytically using $integrateAnalyticalScalar$ with $f(\vec{r}) = 1$. For mismatching dimensions, the integration is done numerically using $integrateNumerical$ with $f(\vec{r}) = 1$.  \\
\begin{mybox}
\begin{lstlisting}
    ctype volume (double tolerance)
    ctype integrateScalar(const LocalPolynomial & P, double tolerance)
    ctype integrateAnalyticalScalar(const LocalPolynomial & P)
    ctype integrateAnalyticalDot(const PolynomialVector & PVec)
    GlobalCoordinate integrateAnalyticalTimes(const PolynomialVector & PVec)

    template <typename Functor>
    ctype integrateNumerical(const Functor & f, double tolerance)
\end{lstlisting}
\end{mybox}



\subsection{Methods - Cached Lagrange Geometry}
\label{interface-curvilineargeometry-cached-methods}



\subsection{Tests}

We start by explicitly defining the local-global mapping functors to mimic LagrangeGeometry for simplices of all 3 dimensions (Table 2). Then the following test procedure is run for each of the simplex mapping:
\begin{itemize}
	\item Loop over 5 interpolatory orders.
	\item For each order sample the interpolatory points from the mapping. Construct LagrangeGeometry and CurvilinearLagrangeGeometry classes.
	\item \textbf{Test 1}. To test if the Geometry has been correctly initialized, return all of its corners and check if they match those evaluated by the function.
	\item \textbf{Test 2}. To test local-to-global functionality, a random set of local points is sampled over the element, and the result of method $global()$ of the Geometry is compared with that of the explicit mapping. The test is omitted if the interpolation order is smaller than the order of the mapping.
	\item \textbf{Test 3}. To test global-to-local functionality, the local method is used for all the interpolation points expecting to obtain the points of the local interpolation grid over reference simplex. Also, it is checked if all these points are reported to be inside the element. This test fails a lot:
		\subitem -Fails to consider the point that is close to the boundary as in.
		\subitem -Fails to converge to a boundary point if the geometry has a zero derivative on the corner or on the whole boundary.
	\item \textbf{Test 4}. To test global-to-local using more probable sample points, the local coordinates are randomly sampled. It is then verified if $\vec{p} \approx local(global(\vec{p}))$. This test also checks if all the sample points are reported inside as they should be.
	\item \textbf{Test 5}. To check if the outside points are correctly sampled outside, as well as check the correctness of surface normal, we loop over all boundaries of the element, construct boundary normals for a uniform grid on the each boundary, and obtain the points which are just outside the element by calculating $\vec{p} = \vec{g} + \alpha \vec{n}$, where $\vec{g}$ is the global coordinate of of some point on the boundary, $\vec{n}$ is the normal at that point, and $\alpha = 0.01$ is a small number. \textbf{[IMAGE]}.
	\item \textbf{Test 6}. The scalar basis functions given in Table 1 are integrated over the Geometry, and results are compared to those calculated by hand, given in Table 2.
	\item \textbf{Test 7}. The dot product surface integrals of vector basis functions (given in Table 3) over the Geometry are compared to those calculated by hand, given in Table 4.
\end{itemize}

\subsection{Integral-tests}

\begin{center}
\begin{tabular}{ | l | l | l |}
  \hline
  Ord & Dim & Scalar Basis Function \\ \hline
  0 & 1 & $1$ \\ \hline
  1 & 1 & $1 + 2x$ \\ \hline
  2 & 1 & $1 + 2x + 3x^2$ \\ \hline
  3 & 1 & $1 + 2x + 3x^2 + 4x^3$ \\ \hline
  4 & 1 & $1 + 2x + 3x^2 + 4x^3 + 5x^4$ \\ \hline
  5 & 1 & $1 + 2x + 3x^2 + 4x^3 + 5x^4 + 6x^5$ \\ \hline
%%  
  0 & 2 & $1$ \\ \hline
  1 & 2 & $1 + 2(x + y)$ \\ \hline
  2 & 2 & $1 + 2(x + y) + 3(x^2 + y^2) + xy$ \\ \hline
  3 & 2 & $1 + 2(x + y) + 3(x^2 + y^2) + xy + 4(x^3 + y^3) + xy^2$ \\ \hline
  4 & 2 & $\begin{array}{lcl} 1 & + & 2(x + y) + 3(x^2 + y^2) + xy + 4(x^3 + y^3) + xy^2 \\ & + & 5(x^4 + y^4) + xy^3 \end{array}$ \\ \hline
  5 & 2 & $\begin{array}{lcl} 1 & + & 2(x + y) + 3(x^2 + y^2) + xy + 4(x^3 + y^3) + xy^2 \\ & + & 5(x^4 + y^4) + xy^3 + 6(x^5 + y^5) + xy^4 \end{array}$ \\ \hline
%%  
  0 & 3 & $1$ \\ \hline
  1 & 3 & $1 + 2(x + y + z)$ \\ \hline
  2 & 3 & $1 + 2(x + y + z) + 3(x^2 + y^2 + z^2) + xy$ \\ \hline
  3 & 3 & $1 + 2(x + y + z) + 3(x^2 + y^2 + z^2) + xy + 4(x^3 + y^3 + z^3) + xyz$ \\ \hline
  4 & 3 & $\begin{array}{lcl} 1 & + & 2(x + y + z) + 3(x^2 + y^2 + z^2) + xy + 4(x^3 + y^3 + z^3) + xyz \\ & + & 5(x^4 + y^4 + z^4) + xyz^2 \end{array}$ \\ \hline
  5 & 3 & $\begin{array}{lcl} 1 & + & 2(x + y + z) + 3(x^2 + y^2 + z^2) + xy + 4(x^3 + y^3 + z^3) + xyz \\ & + & 5(x^4 + y^4 + z^4) + xyz^2 + 6(x^5 + y^5 + z^5) + xyz^3 \end{array}$ \\ \hline
\end{tabular}
\vfill
\title{Table 1. Scalar basis functions used, one of each polynomial order, one per geometry dimension}
\end{center}

\noindent
Using the vector basis functions $(x, x)$ for 2D edges and $(x, y, xy)$ for 3D triangles, we obtain the following integrals for the curvilinear maps

\begin{center}
\begin{tabular}{ | l | l | l | l | l | l | }
  \hline
  mydim & cdim & map               & Normal                  & Integrand                  & Result     \\ \hline
  1     & 2    & $(x,0)$           & $(0,-1)$                & $-x$                       & $-1/2$     \\ \hline
  1     & 2    & $(2x,3x)$         & $(3,-2)$                & $x$                        & $1/2$      \\ \hline
  1     & 2    & $(x,x^2)$         & $(2x,-1)$               & $2x^2-x$                   & $1/6$      \\ \hline
  2     & 3    & $(x,y,0)$         & $(0,0,-1)$              & $-xy$                      & $-1/24$    \\ \hline
  2     & 3    & $(y,3x,x+y)$      & $(-3,-1,3)$             & $-3x-y+3xy$                & $-13/24$   \\ \hline
  2     & 3    & $(y^2,x^2,xy)$    & $(-2y^2,-2x^2,4xy)$     & $-2x^3-2y^3+4x^2y^2$      & $-17/180$   \\ \hline
\end{tabular}
\\
\title{Table 3. DotProduct Integrals of Vector basis functions over curved boundaries}
\end{center}


\begin{landscape}

\begin{center}
\begin{tabular}{ | l | l | l | l | l | l | l | l | l |}
  \hline
  $d_e$ & Map & $\mu(\vec{r})$ & $I_0$ & $I_1$ & $I_2$ & $I_3$ & $I_4$ & $I_5$ \\ \hline
  1 & $(x)$                & $1$ & $1.0$ & $2.0$ & $3.0$ & $4.0$ & $5.0$ & $6.0$ \\ \hline
  1 & $(x,0)$              & $1$ & $1.0$ & $2.0$ & $3.0$ & $4.0$ & $5.0$ & $6.0$ \\ \hline
  1 & $(x,0,0)$            & $1$ & $1.0$ & $2.0$ & $3.0$ & $4.0$ & $5.0$ & $6.0$ \\ \hline
  1 & $(1+2x)$             & $2$ & $2.0$ & $4.0$ & $6.0$ & $8.0$ & $10.0$ & $12.0$ \\ \hline
  1 & $(2x,3x)$            & $\sqrt{13}$ & $\sqrt{13}$ & $2\sqrt{13}$ & $3\sqrt{13}$ & $4\sqrt{13}$ & $5\sqrt{13}$ & $6\sqrt{13}$ \\ \hline
  1 & $(2x,0.5+3x,5x)$     & $\sqrt{38}$ & $1\sqrt{38}$ & $2\sqrt{38}$ & $3\sqrt{38}$ & $4\sqrt{38}$ & $5\sqrt{38}$ & $6\sqrt{38}$ \\ \hline
  1 & $(x^2)$              & $2x$ & $1.0$ & $7/3$ & $23/6$ & $163/30$ & $71/10$ & $617/70$ \\ \hline
  1 & $(x,x^2)$            & $\sqrt{1 + 4x^2}$ & $1.47894286$ & $3.175666172$ & $4.994678155$ & $6.89140143$ & $8.84167808$ & $10.83102449$ \\ \hline
  1 & $(x,x^2,2)$          & $\sqrt{1 + 4x^2}$ & $1.47894286$ & $3.175666172$ & $4.994678155$ & $6.89140143$ & $8.84167808$ & $10.83102449$ \\ \hline
  2 & $(x,y)$              & $1$ & $1/2$ & $7/6$ & $41/24$ & $17/8$ & $37/15$ & $2.75714$ \\ \hline
  2 & $(x,y,0)$            & $1$ & $1/2$ & $7/6$ & $41/24$ & $17/8$ & $37/15$ & $2.75714$ \\ \hline
  2 & $(1+x,x+y)$          & $1$ & $1/2$ & $7/6$ & $41/24$ & $17/8$ & $37/15$ & $2.75714$ \\ \hline
  2 & $(y,3x,x+y)$         & $\sqrt{19}$ & $\sqrt{19}/2$ & $7\sqrt{19}/6$ & $41\sqrt{19}/24$ & $17\sqrt{19}/8$ & $37\sqrt{19}/15$ & $2.75714 \sqrt{19}$ \\ \hline
  2 & $(x^2,y^2)$          & $4xy$ & $1/6$ & $13/30$ & $59/90$ & $103/126$ & $0.94127$ & $1.03915$ \\ \hline
  2 & $(x^2,y^2,xy)$       & $2\sqrt{x^4+y^4+4x^2 y^2}$ & $0.360858$ & $0.938231$ & $1.47326$ & $1.93004$ & $2.33506$ & $2.70079$ \\ \hline
  3 & $(x,y,z)$            & $1$ & $1.0/6$ & $5.0/12$ & $23.0/40$ & $0.676389$ & $0.748214$ & $0.801935$ \\ \hline
  3 & $(x+y,y+z,x+z)$      & $2$ & $1.0/3$ & $5.0/6$ & $23.0/20$ & $2\cdot 0.676389$ & $2\cdot 0.748214$ & $2\cdot 0.801935$ \\ \hline
  3 & $(x^2,y^2,z^2)$      & $8xyz$ & $1.0/90$ & $0.0301587$ & $0.0416667$ & $0.0481922$ & $0.0522134$ & $0.05483$ \\ \hline
\end{tabular} \vfill
\title{Table 2. Explicit mappings for element curvatures, and the integrals of B.F. from Table 1}
\end{center}

\end{landscape}





\chapter{Interface - Curvilinear Grid}

%%%%%%%%%%%%%%%%%%%%%%%%%%%%%%%%%%%%%%%
% Interface of Curvilinear Grid Factory
%%%%%%%%%%%%%%%%%%%%%%%%%%%%%%%%%%%%%%%
\section{Curvilinear Grid}
\label{interface-curvilineargrid}

This section will discuss the new methods available to the curvilinear grid, which extend the dune-grid interface.

\section{Curvilinear Grid Base Implementation}

%%%%%%%%%%%%%%%%%%%%%%%%%%%%%%%%%%%%%%%
% Interface of Curvilinear Grid Factory
%%%%%%%%%%%%%%%%%%%%%%%%%%%%%%%%%%%%%%%
\section{Curvilinear Grid Factory}
\label{interface-grid-factory}

This section will discuss the information that needs to be provided in order to construct a curvilinear grid. \\

\begin{mybox}
\begin{lstlisting}
  Dune::CurvilinearGridFactory<GridType> factory(withGhostElements, verbose, processVerbose, mpihelper);
\end{lstlisting}
\end{mybox}

\noindent
A vertex must be inserted using its coordinate and a global index. It is not possible to insert a vertex without knowing its global index. All vertices belonging to this process must be inserted this way. \\

\begin{mybox}
\begin{lstlisting}
  insertVertex ( const VertexCoordinate &pos, const GlobalIndexType globalIndex )
\end{lstlisting}
\end{mybox}

\noindent
A curvilinear element must be inserted using its geometry type, interpolatory vertex local index vector, interpolatory order and physical tag. Currently only 3D simplex elements are supported. All elements present on this process must be inserted. One should not insert elements not present on this process. The local index of an interpolatory vertex corresponds to the order the vertices were inserted into the grid. The order in which the vertices appear within the vector is according to the dune convention discussed in section \ref{impl-gmsh-numbering-convention}. Currently available interpolation orders are 1-5. The interpolation order must correspond to the number of interpolatory vertices. Currently, physical tag is an integer corresponding to the material property of the entity or otherwise.   \\

\begin{mybox}
\begin{lstlisting}
  void insertElement(GeometryType &geometry, const std::vector< LocalIndexType > &vertexIndexSet, const InterpolatoryOrderType elemOrder, const PhysicalTagType physicalTag)
\end{lstlisting}
\end{mybox}


\noindent
A curvilinear boundary segment must be inserted using its geometry type, interpolatory vertex local index vector, interpolatory order, associated element local index and physical tag. Currently only 2D simplex boundary segments are supported. Currently all boundary segments present on this process must be inserted. One should not insert boundary segments not present on this process. Associated element index is the index of the element neighbouring this boundary segment, its insertion index. In future this parameter requirement may be lifted.     \\

\begin{mybox}
\begin{lstlisting}
  void insertBoundarySegment(GeometryType &geometry, const std::vector< LocalIndexType > &vertexIndexSet, const InterpolatoryOrderType elemOrder, const LocalIndexType associatedElementIndex, const PhysicalTagType physicalTag)
\end{lstlisting}
\end{mybox}


\noindent
Same as standard dune factories, after the creation of a grid a pointer to that grid is returned. It is the duty of the user to delete the grid before the end of program.

\begin{mybox}
\begin{lstlisting}
  GridType * createGrid()
\end{lstlisting}
\end{mybox}



%%%%%%%%%%%%%%%%%%%%%%%%%%%%%%%%%%%%%%%
% Interface of Curvilinear GMSH Reader
%%%%%%%%%%%%%%%%%%%%%%%%%%%%%%%%%%%%%%%
\subsection{Curvilinear GMSH Reader}
\label{interface-gmsh-reader}

This section will precisely discuss the access to the factory and GMSH Reader

\begin{mybox}
\begin{lstlisting}
    Dune::CurvilinearGmshReader< GridType >::read(factory, filename, mpihelper, verbose, processVerbose, writeReaderVTKFile, insertBoundarySegment); 
\end{lstlisting}
\end{mybox}


%%%%%%%%%%%%%%%%%%%%%%%%%%%%%%%%%%%%%%%
% Interface of Curvilinear VTK Writer
%%%%%%%%%%%%%%%%%%%%%%%%%%%%%%%%%%%%%%%
\subsection{Curvilinear VTK Writer}
\label{interface-vtk-writer}


This section will precisely discuss the VTK writer

\begin{mybox}
\begin{lstlisting}
  Dune::CurvilinearVTKWriter<GridType> vtkCurvWriter(verbose, processVerbose, mpihelper);
\end{lstlisting}
\end{mybox}

\begin{mybox}
\begin{lstlisting}
  std::vector<int> tags  { physicalTag, structType, mpihelper.rank() };
\end{lstlisting}
\end{mybox}

\begin{mybox}
\begin{lstlisting}
  vtkCurvWriter.template addCurvilinearElement<mydim>(gt, interpVertices, tags, interpOrder, N_DISCRETIZATION_POINTS, interpolate, explode, WRITE_VTK_EDGES, WRITE_VTK_TRIANGLES);
\end{lstlisting}
\end{mybox}


%%%%%%%%%%%%%%%%%%%%%%%%%%%%%%%%%%%%%%%
% Interface of AllCommunication 
%%%%%%%%%%%%%%%%%%%%%%%%%%%%%%%%%%%%%%%
\include{manual-interface-allcommunicate}







\chapter{Theory}

%%%%%%%%%%%%%%%%%%%%%%%%%%%%%%%%%%%%%%%
% Theory for Lagrange Polynomials
%%%%%%%%%%%%%%%%%%%%%%%%%%%%%%%%%%%%%%%
\section{Lagrange Polynomial Interpolation}
\label{theory-lagrange}

Below we present the theory of interpolation using Lagrange Polynomials, applied to simplex geometries, This section is inspired by \textbf{[wiki-lagrange], [Papers]}, and is a summary of well-known results. The goal is to construct a mapping $\vec{x} = \vec{p}(\vec{r})$ from local coordinates of an entity to global coordinates of the domain. In its own local coordinates, the entity is called \citeDune{} a reference element. A simplex reference element is given by the following coordinates: \\

\noindent
\begin{tabular}{l l l}
\hline
  Label & Dimension & Coordinates \\ \hline
  $\Delta_0$ & 0 & $\{ \}$ \\
  $\Delta_1$ & 1 & $\{ 0\}, \{ 1\}$ \\
  $\Delta_2$ & 2 & $\{ 0, 0 \}, \{ 1, 0 \}, \{ 0, 1 \}$ \\
  $\Delta_3$ & 3 & $\{ 0, 0, 0 \}, \{ 1, 0, 0 \}, \{ 0, 1, 0 \}, \{ 0, 0, 1 \}$ \\
\end{tabular} \\

\noindent
Local simplex geometries can be parametrized using the local coordinate vector $\vec{r}$: \\

\noindent
\begin{tabular}{l l l}
\hline
  Entity      & Parametrization    & Range \\ \hline
  Edge        & $\vec{r}=(u)$      & $u \in [0,1]$ \\
  Triangle    & $\vec{r}=(u,v)$    & $u \in [0,1]$ and $v \in [0, 1-u]$ \\
  Tetrahedron & $\vec{r}=(u,v,w)$  & $u \in [0,1]$, $v \in [0, 1-u]$ and $w \in [0, 1-u-v]$ \\
\end{tabular} \\

\subsection{Interpolatory Vertices}
\label{theory-lagrange-vertices}

\noindent
In order to define the curvilinear geometry, a set of real geometry points $\vec{x}_i = \vec{p}_i(\vec{r}_i)$ is given to be interpolated over. By convention, the real geometry is sampled over a structured grid on a reference simplex, namely
\[\vec{r}_{i,j,k} = \frac{(i,j,k)}{Ord}, \;\;\; i=[0..Ord], \;\;\; j=[0..Ord-i], \;\;\; k=[0..Ord-i-j]\]
where $Ord$ is the interpolation order of the surface. Thus, points from this uniform grid in local coordinates must be mapped to the provided points in global coordinates. It is the job of the meshing software (e.g. GMSH\citeGMSH{}) to ensure that the global geometry of an entity is non-singular / non-self-intersecting. In principle, a non-uniform grid could be used in order to minimize the effect of Runge phenomenon \textbf{CITE\_RUNGE}, However, it is not an issue for lower dimensions. \\

\noindent
One can verify that the above discretization generates the following number of interpolatory points: \\

\noindent
\begin{tabular}{l l l l l l}
\hline
  Entity \textbackslash Order & 1 & 2  & 3  & 4  & 5 \\ \hline
  Edge                        & 2 & 3  & 4  & 5  & 6 \\
  Triangle                    & 3 & 6  & 10 & 15 & 21 \\
  Tetrahedron                 & 4 & 10 & 20 & 35 & 56 \\
\end{tabular} \\


\subsection{Interpolatory Polynomials}
\label{theory-lagrange-polynomials}

\noindent
The number of interpolatory points above exactly matches the total number of monomials necessary to construct a complete polynomial of order $Ord$ or less. It is quite obvious, since the above discretization matches the binomial/trinomial triangle. We define the functions $z^{(1,i)}(u)$, $z^{(2,i)}(u,v)$ and $z^{(3,i)}(u,v,w)$ as the set of all monomials of corresponding order, where the first parameter is dimension of the entity, and the 2nd is the polynomial order:
\begin{itemize}
	\item edge: \\
		$z^{(1,1)}(u) = \{1, u\}$, \\
		$z^{(1,2)}(u) = \{1, u, u^2\}$, \\
		$z^{(1,3)}(u) = \{1, u, u^2, u^3\}$, \\
		$z^{(1,4)}(u) = \{1, u, u^2, u^3, u^4\}$, \\
		$z^{(1,5)}(u) = \{1, u, u^2, u^3, u^4, u^5\}$, \\
		etc
	\item face:	\\
		$z^{(2,1)}(u,v)	= \{1, u, v\}$, \\
		$z^{(2,2)}(u,v) = \{1, u, v, u^2, uv, v^2\}$, \\
		etc
	\item tetrahedron: \\
		$z^{(2,1)}(u,v,w) = \{1, u, v, w\}$, \\ 
		$z^{(2,2)}(u,v,w) = \{1, u, v, w, u^2, uv, v^2, wu, wv, w^2\}$, \\
		etc
\end{itemize}

\noindent
Ultimately, we wish to approximate the mapping $\vec{p}(\vec{r})$ by a polynomial of the order $Ord$, such that it exactly fits the provided interpolatory vertices $\vec{x}_i$. Since there are as many interpolatory vertices as there are monomials, such a polynomial will be unique. This allows the simplex geometries to be interpolated by \textit{complete} basis of a given order. This is not the same for all entities. For example, for hexahedra, these numbers do not match, therefore one either has to use interpolation of incomplete polynomial order, or choose a more sophisticated local discretization. \textbf{[Volakis2010]} choose the first approach, interpolating a 9 node 2nd order rectangle with 4th order incomplete polynomial which has a convenient separable tensor product form. \\

\noindent
But back to simplices. One way to write such a polynomial approximation is
\begin{equation}
	\vec{p}(\vec{r}) = \sum_j L_j(\vec{r})\vec{p}_j 
\end{equation}
\noindent
where the Lagrange Polynomials $L_j$ are defined by their interpolatory property
\begin{equation}
	\label{equation-lagrangepol-interpolatory-property}
	L_j(\vec{r}_i) = \delta_{ij}
\end{equation}
\noindent
for all interpolatory points $\vec{r}_i$. The advantage of this formulation is that the Lagrange Polynomials are independent of the exact coordinates ($\vec{x}_i$), and thus can be pre-computed and reused for all entities of a given order. \\

\noindent
It remains to determine the exact form of Lagrange Polynomials. We would like to prove that the following equation holds:
\begin{equation}
	\label{equation-lagrangepol-basis-link}
	z_i(\vec{r}) = \sum_j L_j(\vec{r}) z_i (\vec{r}_j) 
\end{equation}
\noindent
where $z_i$ is a vector of monomials defined above. This equation should hold for all $z^{(\dim, Ord)}$, where $\dim = \{1,2,3\}$. For $\dim < 3$ where $z$ is defined for less than 3 parameters, simply ignore the extra parameters in $\vec{r}$. \\

\noindent
The proof is quite simple. Both LHS and RHS are polynomials of order at most $Ord$, which means that they have at most $N_{Ord}$ free parameters, and therefore, if we can show that the equation holds for $N_{Ord}$ different parameter sets, then it holds for all others as well. And indeed, \eqref{equation-lagrangepol-basis-link} holds for all $\vec{r} = \vec{r}_k$ because of \eqref{equation-lagrangepol-interpolatory-property}. \\

\noindent
Finally, we can write \eqref{equation-lagrangepol-basis-link} as a vector equation
\begin{equation}
	\vec{z} (\vec{r}) = V \vec{L} (\vec{r})
\end{equation}
\noindent
where $V_{ij} = z_i (\vec{r}_j)$, and find the Lagrange polynomials by inverting $V$, namely
\begin{equation}
	\vec{L} (\vec{r}) = V^{-1} \vec{z} (\vec{r})
\end{equation}

\noindent
It is important to understand that the resulting interpolated geometry in global coordinates is NOT exhaustively defined by the shape of its boundary, as the geometry inside the entity also undergoes this polynomial transformation.


\subsection{Implementation for Simplices}
\label{subsection-simplexgrid}

\noindent
In this section we discuss how to efficiently enumerate the simplex interpolatory points, and to construct the reference simplex grid. \\

\noindent
Let us place a set of points $\vec{\eta} \in Z^{\dim}$ over simplex $\Delta^{\dim}_{\mathrm{len}}$. This can be done trivially
by using 3 for loops and pushing vectors into a vector
\begin{itemize}
	\item $\Delta^{1}_n = \{(i)\}$, for $i = [1$ to $n]$
	\item $\Delta^{2}_n = \{(j,i)\}$, for $i = [1$ to $n]$, $j = [1$ to $n - i]$
	\item $\Delta^{3}_n = \{(k,j,i)\}$, for $i = [1$ to $n]$, $j = [1$ to $n - i]$, $k = [1$ to $n - i - j]$
\end{itemize}

\noindent
Then, each point $(\Delta^{d}_n)_i$ corresponds exactly to the power of $u,v,w$ in the expansion of $(1 + u + v + w)^n$, namely
\[ (1 + u)^n = \sum_{i=0}^n C^{(\Delta^{1}_n)_i}_n u^{(\Delta^{1}_n)_{i,1}} \]
\[ (1 + u + v)^n = \sum_{i=0}^n C^{(\Delta^{1}_n)_i}_n u^{(\Delta^{1}_n)_{i,1}} v^{(\Delta^{1}_n)_{i,2}} \]
\[ (1 + u + v + w)^n = \sum_{i=0}^n C^{(\Delta^{1}_n)_i}_n u^{(\Delta^{1}_n)_{i,1}} v^{(\Delta^{1}_n)_{i,2}} w^{(\Delta^{1}_n)_{i,3}} \]

\noindent
where $C^{i}_n, C^{i,j}_n$ and $C^{i,j,k}_n$ are the binomial, trinomial and quatranomial coefficients. The powers of the parameters given in this way are exactly the complete monomial basis for a polynomial of order up to and including $d$. \\

\noindent
Also, it is convenient to note that $(\Delta^{d}_n)_i / n$ is exactly the parametric coordinates of the interpolation points on a regular grid over simplex. 

\begin{figure}[hp]
    \centering
    \includegraphics[scale=0.5]{doc-pics/pic-simplex-grid.png}
    %\caption{Awesome Image}
    %\label{fig:awesome_image}
\end{figure}

\noindent
After the monomials and the parametric interpolation points have been constructed, it remains to construct the interpolation matrix by evaluating the monomials at the interpolation points, then to invert the matrix, and multiply the monomial vector by it obtaining lagrange polynomials. This has been implemented both explicitly, by calculating all the lagrange interpolatory polynomials for simplices and writing them as functions, and implicitly, by introducing a polynomial class, which has all the above functionality, and thus generates a set of interpolatory polynomials which can be evaluated and integrated analytically by the code.


%%%%%%%%%%%%%%%%%%%%%%%%%%%%%%%%%%%%%%%
% Theory for global and local mappings
%%%%%%%%%%%%%%%%%%%%%%%%%%%%%%%%%%%%%%%
\subsection{Coordinate transformation}

In order to calculate the coordinate transformation properties, one requires the knowledge of the local-to-global map $\vec{p}(\vec{r})$ and its first partial derivatives. Currently, only the Lagrange Polynomials themselves are provided as hard-coded expressions, but their derivatives are not yet available as hard-coded quantities, and thus are extracted from the analytical map. \\

\noindent
\textbf{Jacobian and JacobianInverse}.
We construct the local-to-global map using the polynomial class, and then compute the Jacobian $J_{ij}(\vec{r}_0) = \partial_{r_i} p_j (\vec{R}) |_{\vec{r}_0}$ using partial differentiation of the polynomial class. This results in a matrix of polynomials, which can be evaluated for the desired local coordinate(s). The inverse and integration element of the element are computed numerically, using Dune-MatrixHelper to invert matrix and calculate the pseudodeterminant $dI = \sqrt{\det(JJ^T)}$. \\

\noindent
\textbf{Local-to-Global mapping}
This is the map that defines the curvature of the element via Lagrange Interpolation. To accelerate the calculation, the local-to-global map is computed numerically using hard-coded Lagrange polynomials \\

\noindent
\textbf{Global-to-Local mapping}
The global-to-local method inverts the above map, and finds the local coordinate (of the entity geometry) $\vec{r}$ that corresponds to the requested global coordinate $\vec{x}$. Further, this method is extended to elements with $(dim_{elem} \leq dim_{world})$ by converting it to an optimization problem
\begin{equation}
  \label{eq-theory-mapping-optimization}
  \vec{r} : |\vec{p}(\vec{r}) - \vec{x} |^2 \rightarrow \min
\end{equation} 
searching for the local coordinate closest to the inverse of the desired global coordinate in terms of distance in global coordinates. \\

\noindent
While this problem is always uniquely solvable in linear case, in curvilinear case it experiences several additional challenges
\begin{itemize}
	\item Polynomial interpolatory map $\vec{p}(\vec{r})$ is strictly bijective inside the reference element, which must be ensured by the mesh generator. However, this need not be the case outside it. For example, $p(r) = r^2$ is a perfectly valid 1D local-to-global map for an edge defined on $[0,1]$. However, the map is clearly non-invertible for all $p(r) \leq 0$.
	\item Curvilinear geometries have singularities, like $r = 0$ in the previous example. At these points Jacobian determinant changes sign, and the volume element is zero. Simple iterative methods break down in the viscinity of singularities. 
	\item For $(dim_{elem} \leq dim_{world})$, the optimization problem \ref{eq-theory-mapping-optimization} is non-convex. It can have multiple solutions, even uncountably many.
\end{itemize}

\noindent
For obvious reasons we will not solve the problem directly, as searching for roots of a system of polynomial equations with several parameters is a very challenging task \cite{canny+1989}. Instead, the problem is solved by a first order Gauss-Newton method \cite{bjoerck+1996}, originally implemented in dune-multilinear geometry.

\noindent
After having discussions with potential users, we have realised that in order to satisfy all use cases there is a need to implement two distinct methods
\begin{itemize}
  \item Restrictive method. This method will be useful for those who want to find the element containing the global coordinate, as well as the local coordinate inside that element. If the provided global coordinate is inside the element, the method will return true and a valid local coordinate. Otherwise, it will simply return false and no coordinate at all. This method also extends to lower dimension entities, finding the local coordinate which optimizes the distance within the element. Given well-defined map, this method is guaranteed fo finish.
  \item Non-restrictive method. This method will be useful for those who wish to extrapolate the global-to-local map beyond the reference element. This method searches for the inverse (or the distance minimizer) over the entire local domain. This is a best effort method - due to the above mentioned difficulties, it will frequently fail to find an acceptible solution. In this case, an exception is thrown.
\end{itemize}

\noindent
Below we briefly present the algorithm of the restrictive method:
\begin{enumerate}
	\item Let $\vec{x}_0$ be the requested global coordinate
	\item Start with a local point $\vec{r}_0$ guaranteed to be inside the element (e.g. its center),
	\item Iteratively choose better approximations for local coordinate using \[\vec{r}_{n+1} = \vec{r}_n + \vec{d}(\vec{r}_n)\] where $\vec{d}(\vec{r}_n)$ is the solution of
	        \[ J(\vec{r}_n) \vec{d}(\vec{r}_n) = \vec{p}(\vec{r}_n) \] and $J(\vec{r})$ is the Jacobian matrix.
	\item We finish the iterative process, if the global coordinate converges in terms of the two-norm
	        \[\Delta_i = \{ |\vec{p}(\vec{r}_i) - \vec{x}_0 |^2 \}. \]
	\item We also terminate the iteration if it is suspected that the vertex is outside the element. For this we use two criteria: the current iteration being far outside the element \[|\vec{p}_0 - \vec{p}_i|_2 > 4 R_{elem}\] and the convergence being slower than expected \[ 2 \Delta_{i + 10} > \Delta_{i} \]
\end{enumerate}

\noindent
We are still looking to improve this method. It correctly predicts the global coordinates being inside and outside the element for most of our tests, but fails to identify the boundary points inside the element for certain cases. We warmly welcome suggestions on this question.


%%%%%%%%%%%%%%%%%%%%%%%%%%%%%%%%%%%%%%%
% Theory for integration over curvilinear entities
%%%%%%%%%%%%%%%%%%%%%%%%%%%%%%%%%%%%%%%
\section{Integration}

It is necessary to integrate scalar and vector functions over an element. \\

\noindent
In original strategy (Multilinear Geometry using Gaussian quadrature), the geometry only needed to provide an integration element, the integration itself was performed in dervied classes outside DUNE. However, for the special case of integrating a polynomial function over the element, the resulting integral is frequently possible to calculate analytically. Such integration requires polynomial functionality, and therefore it is more comfortable to implement within DUNE, than to re-implement by the user every time.


\subsection{Overview of available numerical methods}

\noindent
Below is presented a short summary of integration method types known to us: \\

\noindent
\textbf{Gaussian Quadrature}: The method available in DUNE.
\begin{itemize}
	\item This method calculates the integral as a linear product of the integrand $f(\vec{r})$ values at specific precomputed points $\vec{r}_i$ with specific precomputed weights $w_i$, namely $I = \sum_i w_i f(\vec{r}_i)$. Thus the main advantage of this method is its computational cost, which is small for low order polynomial integrands.
	\item Optimal quadrature points are only available for small dimensions. Finding such point sets for high dimension polynomials is very involved and is known to suffer from finite precision of floating point arithmetic. Alternatively, a suboptimal point distribution can be obtained from a tensor product space of 1D point distributions, whose size grows exponentially with integration dimension.
	\item Gaussian Quadrature is constructed with the idea of calculating exact integrals for integrands being polynomial up to a given order. However, when integrating over a curved boundary, thte integration elemen is a square root of a polynomial, and polynomials really badly approximate square root, especially for small arguments, which can easily happen for highly curved elements. Not to mention that one, in principle, could with to integrate arbitrary (within reason) functions over the element. Thus
		\subitem - Can GQ estimate integration error for non-polynomial functions?
		\subitem - Can it be made hierarchical to have control over error by refinement?
		\subitem - What would be the convergence rate to compare with other methods?
\end{itemize}

\noindent
\textbf{Interpolatory adaptive refinement}: The method currently implemented in LagrangeGeometry subclass
\begin{itemize}
	\item - Evaluates integral over element, approximating the integrand by an interpolatory polymomials of two hierarchical orders (2 and 4 at the moment) \textbf{[IMAGE HERE]}
	\item - The running integration error is approximated by the difference between the analytical integrals calculated from these two interpolatory polynomials.
	\item - The higher order element is split into into sub-elements of lower order, and the integration proceeds recursively.
	\item - Every time an element is split, its previous running error is subtracted from the total error, and the running errors of the sub-elements are added to the total error. Thus, the integration is terminated when total approximated error is below selected tolerance. 	
		\subitem - using heap structure ordered by the approximate error of the element. This way avoids recursion, and at every iteration selects the element which has worst error, then refines it.
		\subitem - When splitting, the previously calculated points are not re-calculated but hierarchically re-used by sub-elements. The sub-element only needs to be refined to a higher hierarchical order, by adding more points.
	\item \textit{Possible improvement - Performance}. As the the refined element does not check if the neighboring elements are also being refined, so they both sample on the boundary twice. Does there exist a method to store/find intersection refinements faster than just compute 2nd time. Using order 4 for every new refined triangle we sample 9 new points, out of which 2 are being wasted, thus $22\%$ inefficient.
\end{itemize}

\noindent
\textbf{Monte-Carlo integration} - according to above mentioned paper, good for dimensions 7 and above.
\begin{itemize}
	\item Randomly samples function over element, integral is approximated by the average over the sample
	\item natural error estimate using sample standard deviation
	\item Stratified Sampling: if after a set number of iterations sample error is larger than expected, then function is highly non-uniform. Split element in equal parts and continue recursively.
	\item Markov Chain Monte Carlo (MCMC): Uses random walk to sample the integrand, thus concentrating the sample points where the function varies most. Can use Metropolis-Hastings to also adapt the sampling distribution.
\end{itemize}

\noindent
\textbf{Interpolatory Spline integration} - this is just another idea...
\begin{itemize}
	\item Makes grid over element, cubic-interpolates all consecutive partially-overlapping segments, integrates analytically over each segment.
		\subitem -tricky part 1: when interpolatory segments intersect, with what weights to take the intersecting parts
		\subitem -tricky part 2: how well does this method interpolate boundaries of the element (being close to edges and faces)
		\subitem -tricky part 3: how to estimate error of integration and necessary grid step?
		\subitem - Any way to do the refinement or error-control?
\end{itemize}




\subsection{Integration Element - Vector}

When integrating vector functions we are mostly interested in the integrals over boundary surfaces and edges, namely $\int_{\partial V} \vec{f}(\vec{r}) \cdot \vec{n}(\vec{r}) d(\partial V)$. For an edge in 2D the following expression for the tangential and normal integration elements (up to a sign convention) can be found:
\[ d\vec{l}_{\parallel} = (\partial_u p_x, \partial_u p_y)du \; \; \; \; \; d\vec{l}_{\perp} = (\partial_u p_y, -\partial_u p_x)du  \]

\noindent
For a vector in 3D the tangential integration element is not defined, but the normal integration element is
\[ d\vec{S} = (\partial_u \vec{p} \times \partial_v \vec{p})du \; dv  \]

\noindent
Thus, given polynomial vector basis functions $\vec{f}$ and polynomial interpolation, the scalar (and, if necessary, vector) products $\vec{f}(u) \cdot d\vec{l}(u)$ and $\vec{f}(u,v) \cdot d\vec{S}(u,v)$ are also polynomial, and can be integrated exactly using analytic polynomial integration code.




\subsection{Integration Element - Scalar}

In its most general form a scalar integral over an element can be written as \[\int f(\vec{r}) d^{\dim} x = \int f(\vec{r}) \mu(\vec{r}) d^{\dim} r,\] where the integration element $\mu(\vec{r})$ can most generally be written as \[\mu(\vec{r}) = \sqrt{\det(J J^T)},\] where $J$ is the Jacobian matrix. \\

\noindent
In the case of matching element and space dimension (like volume in 3D, and area in 2D), the integration element simplifies to $|\det J|$. Even though absolute value is not a polynomial function, it can be observed that $\det J$ is not allowed to change sign over the element for any sensible non-self-intersecting geometries. Even if $\det J = 0$ somewhere within the element, it would mean that a finite volume element inside reference element is mapped to a 0 volume in real space, which should be avoided by GMSH at the stage of selecting interpolation points. Therefore, in this case, $\det J$ will always have the same sign. It remains to evaluate it anywhere inside the element, and multiply by -1 if it is negative. Then the integration of scalar polynomial function over this integration element is done analytically. \\

\noindent
In the case of mismatching dimensions (like area in 3D, and length in 2D and 3D), the expression for $\mu(\vec{r})$ can not be simplified, and therefore is an integral of the form \[\int f(\vec{r}) \sqrt{g(\vec{r})} d(\partial V),\] where $f$ and $g$ are polynomials. This integral cannot be done analytically, and therefore requires a numerical method.

\noindent
According to \textbf{[PAPER]} best results in low-dimensional numerical integration are achieved by adaptive quadrature of high degree, whereas Monte-Carlo methods are best for high-dimensional integrals. I propose to integrate into Dune one of the opensource adaptive quadrature routines, for example the GSL extension from \url{http://ab-initio.mit.edu/wiki/index.php/Cubature}. These are based on Clenshaw–Curtis quadrature, which has the advantage of being Hierarchical, and thus can be refined to iteratively improve precision withouth having to sacrifice previous sample points. \\

\noindent
At the moment we have constructed an adaptive interpolation integrator class within DUNE to deal with calculating curved edge lengths and face areas, when their dimension is smaller than the world dimension. If we want to provide such information, having an integrated numerical integrator is inevitable.


%%%%%%%%%%%%%%%%%%%%%%%%%%%%%%%%%%%%%%%
% Theory for point location (OCTree)
%%%%%%%%%%%%%%%%%%%%%%%%%%%%%%%%%%%%%%%
\section{Implementation Details - OCTree}

\subsection{Point location with respect to a plane}

Assume that a plane is given by 3 coordinates $\vec{a}, \vec{b}, \vec{c}$, and we would like to check on which side of this plane the point $\vec{p}$ is. This is uniquely given by a function
\begin{equation}
\label{equation-point-plane-test}
	ptest(\vec{a}, \vec{b}, \vec{c}, \vec{p}) = \mathrm{sgn} \{ \det (\vec{a} - \vec{p}, \vec{b} - \vec{p}, \vec{c} - \vec{p}) \}
\end{equation}
\noindent
This function takes values $1,-1$ for corresponding sides of the plane, and $0$ if the point is on the plane.

\subsection{Locating the element the given global coordinate belongs to}
\label{subsection-locating-element}

\noindent
Current convention: loop over all elements in the mesh, check if the point is inside the element using the method below. \\


\noindent
This method has $O(1)$ initialization cost, but $O(N) * ins$ query cost, which becomes wastefully slow as the number of points to locate grows. Here $ins$ is the cost of $is\_inside$ method, which has constant complexity if the query point is far away from the triangle, but requires an iterative method for nearby methods.

\noindent
There are two ways one can implement the above idea: \\
\textbf{First method:}
\begin{enumerate}
	\item Iterate over all elements, use linear $check\_inside()$
	\item For the element for which linear $check\_inside() = true$ run the nonlinear $check\_inside()$
	\item If the nonlinear $check\_inside() = false$, recursively do nonlinear $check\_inside()$ on the neighbors of the element.
\end{enumerate}
\textbf{Second method:}
\begin{enumerate}
	\item Iterate over all elements, use nonlinear $check\_inside()$
\end{enumerate}
\noindent
The first method is better, because it will on average have less runs of the iterative method, as if the point is located in a linear element, it has high probability of being located inside the nonlinear element with the same corners. \\

\noindent
An improvement would be to implement an OCTree, which will have a recursively-refined cubic mesh over the tetrahedral one, and upon initialization would place each of the tetrahedrons recursively within one or more leafs of the tree. Thus the initialization cost would be $O(N \log(N))$, and the cost of a single query $O(\log(N)) + M * inc$, where $M$ is the number of neighbors an element has. This is a considerable improvement for large number of queries. \\

\noindent
\textbf{OCTree method}:
\begin{enumerate}
	\item Construct the OCTree - grid the domain, build the hierarchical tree, place the correct element labels into each leaf
	\item For each query, locate the leaf the query point corresponds to
	\item Proceed by using one of the above methods on the element set of the leaf
\end{enumerate}

\begin{figure}[hp]
    \centering
    \includegraphics[scale=0.7]{doc-pics/pic-octree-3.png}
	\caption{OcTree internal structure}    
    
    \includegraphics[scale=0.7]{doc-pics/pic-octree-1.png}
	\caption{OcTree grid}    
	
    \includegraphics[scale=0.7]{doc-pics/pic-octree-2.png}
	\caption{OcTree partition of elements}    
    %\caption{Awesome Image}
    %\label{fig:awesome_image}
\end{figure}


\noindent
How does this work with parallel processing?

\subsection{Checking if a global coordinate is inside a given element (straight sided)}
\label{subsection-isinside-linear}

\noindent
For straight-sided elements the conversion between global and local coordinates is 1-to-1 in the whole space.
Therefore, it is sufficient to find the corresponding local coordinate and use the $referenceelement$ method $checkInside$. \\

\noindent
For example, for a simplex the $is\_inside(u,v,w)$ for local coordinates $(u,v,w)$ looks like 
\[u \geq 0\ \&\ v \geq 0\ \&\ w \geq 0,\ \&\ u+v+w \leq 0. \]

\noindent
Naturally, the global-local map has a finite precision, therefore the $checkInside$ method corrects for that by having a small tolerance for above inequalities, thus avoiding the case where a boundary point would be considered outside both neighboring elements because of numerical errors.


\subsection{Checking if a global coordinate is inside a given element (curvilinear elements)}
\label{subsection-isinside-nonlinear}

\noindent
This method is only defined if ($(dim_{elem} = dim_{world})$). For discussion see the $local()$ method discussion.

\noindent
In principle, this method requires an iterative solver, but first we run two simpler tests which immediately identify some of the points inside or outside of the element. \\

\noindent
\textbf{Far-point test}:
\begin{enumerate}
	\item Define linear center $\vec{p}_{CoM}$ of an element as the center of mass of its corners.
	\item Define the radius of an element $R$ as the largest distance between $\vec{p}_{CoM}$ and a point $\vec{p}_b$ on its boundary
	\item Define linear radius $R_{lin}$ to be the largest distance between $\vec{p}_{CoM}$ and one of the corners of the element. It can be shown that $R_{lin} = \frac{\sqrt{dim^2 + dim - 1}}{dim + 1} $, which is $\{ \frac{1}{2}, \frac{\sqrt{5}}{3}, \frac{\sqrt{11}}{4} \}$ for dimensions $\{1,2,3\}$.
	\item Demand that for all sensible curvilinear elements, $R$ should be bounded by some scaling of $R_{lin}$. For example, we can require that $R \leq 2 R_{lin}$, which would mean that all poins of every element should be entirely contained within $2 R_{lin}$ of its center. A more precise prefactor could be calculated from curvature constants of the bounaries of the element (max over the boundary of some expr. involving derivatives of interpolatory polynomials).
	\item Thus, if $|\vec{p}_{CoM} - \vec{p}|_2 > 2 R_{lin}$, we can immediately report that $\vec{p}$ is outside the element.
\end{enumerate}

\noindent
\textbf{Global Barycentric Coordinate test}:
\begin{enumerate}
	\item Define a global barycentric coordinate as the area enclosed by one curved boundary of the element, the point of interest, and the straight-sided boundaries that connect the point of interest and the corners of the curved boundary.
		\subitem - for 2D triangle, a barycentric coordinate is given by
		\[B = \frac{1}{2}\int_0^1 (\vec{p}(u) - \vec{p}_0) \times \partial_u \vec{p}(u) du\]
		as derived from triangle area $S = \frac{1}{2} \vec{a} \times \vec{b}$. Here $\vec{p}_0$ is the coordinate of interest.
		\subitem - for 3D tetrahedron, a barycentric coordinate is given by
		\[B = \frac{1}{3}\int_0^1 \int_0^{1-u} (\vec{p}(u,v) - \vec{p}_0) \cdot (\partial_u \vec{p}(u,v) \times \partial_v \vec{p}(u,v)) du\]
		as derived from tetrahedron volume area $V = \frac{1}{6} \vec{a} \cdot (\vec{b} \times \vec{c})$.
		Note that there is an additional factor of 2 in the barycentric equation, because triangular grid only covers half of the triangle, whereas
		parallelogram grid covers the whole (check figure \ref{fig:barycentric_coordinate_calculation})

\begin{figure}[!htb]
    \centering	
    \includegraphics[scale=0.7]{doc-pics/pic-barycentric-curvilinear-definition.png}
    \caption{Definition of Barycentric Curvilinear Coordinates}
    %\label{fig:awesome_image}
\end{figure}

	\item For linear elements, the sum of barycentric coordinates always equals the total volume of the element for internal points, and is larger than that for external points. This is not true for non-convex elements, as the sum may be larger than the volume even for internal points. The method can not be improved by considering the sign of the barycentric coordinate based on the orientation of the boundary, as it is the same for internal and external points of the concave surface.
	\item Thus, if the sum of global barycentric coordinates is equal to the volume of the element, the point is automatically inside the element, otherwise we remain uncertain.
			
\end{enumerate}

\begin{figure}[!htb]
    \centering	
    \includegraphics[scale=0.7]{doc-pics/pic-barycentric-problem-1.png}
    \includegraphics[scale=0.7]{doc-pics/pic-barycentric-problem-2.png}
    \includegraphics[scale=0.7]{doc-pics/pic-barycentric-problem-3.png}
    \caption{Problem cases with Barycentric Curvilinear coordinates}
    %\label{fig:awesome_image}
\end{figure}

\noindent
If both the above tests do not give a conclusive result, we need to use Global-To-Local mapping, and then $checkInside$ for the local coordinate. The challenges of this approach are described in the corresponding section. \\

\noindent
\textbf{Current Implementation: } There is no $is\_inside()$ method, only $local()$ method. Local method returns false if the point is not inside the element, and returns true and the correct local coordinate if it is. If the answer is true, we have to calculate the local coordinate anyway to make sure, makes sense to return it not to calculate it twice.

\begin{figure}[p]
    \centering
    \includegraphics[scale=0.7]{doc-pics/pic-bounded-curvilinear-integral.png}
    \caption{Barycentric Coordinate Calculation}
    \label{fig:barycentric_coordinate_calculation}
\end{figure}






\chapter{Implementation Details}

%%%%%%%%%%%%%%%%%%%%%%%%%%%%%%%%%%%%%%%%%%%%%%%%%%%%%%%%%%%%%%%%%%%%%
% Implementation Details - Curvilinear Element Interpolator
%%%%%%%%%%%%%%%%%%%%%%%%%%%%%%%%%%%%%%%%%%%%%%%%%%%%%%%%%%%%%%%%%%%%%

\section{Implementation Details - Curvilinear Geometry}

\subsection{Local-to-global Mapping}

\subsection{Normals}

\subsection{Analytical Integration}


%%%%%%%%%%%%%%%%%%%%%%%%%%%%%%%%%%%%%%%%%%%%%%%%%%%%%%%%%%%%%%%%%%%%%
% Implementation Details - Curvilinear GMSH Reader
%%%%%%%%%%%%%%%%%%%%%%%%%%%%%%%%%%%%%%%%%%%%%%%%%%%%%%%%%%%%%%%%%%%%%


\section{Implementation Details - Curvilinear GMSH Reader}

\subsection{Structure of .msh files}

\begin{mybox}
\begin{lstlisting}
$MeshFormat
ver f_type data_size    # This line is mostly irrelevant
$EndMeshFormat
$Nodes
n_vertices
1 x y z
2 x y z
.......
n_vertices x y z
$EndNodes
$Elements
n_elem
1 elem_type n_tags (process_tags) v_1 v_2 ... v_N
2 elem_type n_tags (process_tags) v_1 v_2 ... v_N
.......
n_elem elem_type n_tags (process_tags) v_1 v_2 ... v_N
$EndElements
\end{lstlisting}
\end{mybox}

\noindent
where
\begin{itemize}
	\item $ver$             - version of the GMSH file
	\item $f\_type$          - type of file (irrelevant)
	\item $data\_size$       - size of file (irrelevant)
	\item $n\_vertices$      - number of vertices of the mesh
	\item $i\ x\ y\ z$         - index of the vertex and its coordinates
	\item $n\_elem$          - number of elements of the mesh
	\item $elem\_type$       - Integer which determines element type and interpolation order
	\item $n\_tags$          - Total number of tags. If $>2$, then have $process\_tags$
	\item $process\_tags$    - Tags which determine the process the vertex belongs to. Only if GMSH is told to partition the mesh
	\item $v\_1\ v\_2\ ...\ v\_N$ - Indices of interpolatory vertices associated with this element (includes corners)
\end{itemize}

\subsection{Convention for numbering interpolatory vertices}
\label{impl-gmsh-numbering-convention}

\noindent
Each curvilinear element posesses a set of interpolatory vertices. For order 1 (linear elements) this is just the set of corners of the element, but for higher orders there are additional points located in on the inside of the elements, their faces and edges. To number these vertices, GMSH uses recursive conviention.

\begin{enumerate}
	\item First number all corners, then all edges, then all faces, then vertices inside element, then internal vertices of the element
	\item Inside edge, vertices are numbered sequentially
	\item For a triangle, the order of edges is $(0,1)$, $(1,2)$, $(2,0)$. (in 2D)
	\item For a tetrahedron, the order of edges is $(0,1)$, $(1,2)$, $(2,0)$, $(3,0)$, $(3,2)$, $(3,1)$.
	\item For a tetrahedron, the order of faces is $(0, 2, 1)$, $(0, 1, 3)$, $(0, 3, 2)$, $(3, 1, 2)$, including orientation
	\item If there are vertices associated with element itself(for example, on the triangle or inside the tetrahedron), a smaller element is created inside this triangle preserving its orientation, and is then numbered recursively.
\end{enumerate}

\noindent
Unfortunately, this convention does not match the grid convention used in DUNE, namely
\begin{mybox}
\begin{lstlisting}
for (z=0 to 1, y=0 to 1-z, x=0 to 1-z-y) { vertex(x,y,z); }
\end{lstlisting}
\end{mybox}

\noindent
There is no good expression which maps from GMSH to DUNE convention, so it was implemented it by hand for simplex geometries up to order 5.
\begin{itemize}
	\item Triangle Order 1: \{0, 1, 2\}
	\item Triangle Order 2: \{0, 3, 1, 5, 4, 2\}
	\item Triangle Order 3: \{0, 3, 4, 1, 8, 9, 5, 7, 6, 2\}
	\item Triangle Order 4: \{0, 3, 4, 5, 1, 11, 12, 13, 6, 10, 14, 7, 9, 8, 2\}
	\item Triangle Order 5: \{0, 3, 4, 5, 6, 1, 14, 15, 18, 16, 7, 13, 20, 19, 8, 12, 17, 9, 11, 10, 2\}
	
	\item Tetrahedron Order 1: \{0, 3, 1, 2\}
	\item Tetrahedron Order 2: \{0, 7, 3, 4, 9, 1, 6, 8, 5, 2\}
	\item Tetrahedron Order 3: \{0, 11, 10, 3, 4, 17, 14, 5, 15, 1, 9, 18, 12, 16, 19, 6, 8, 13, 7, 2\}
	\item Tetrahedron Order 4: \{0, 15, 14, 13, 3, 4, 25, 27, 19, 5, 26, 20, 6, 21, 1, 12, 28, 29, 16, 22, 34, 31, 24, 32, 7, 11, 30, 17, 23, 33, 8, 10, 18, 9, 2\}
	\item Tetrahedron Order 5: \{0, 19, 18, 17, 16, 3, 4, 34, 39, 36, 24, 5, 37, 38, 25, 6, 35, 26, 7, 27, 1, 15, 40, 43, 41, 20, 28, 52, 55, 46, 33, 53, 49, 30, 47, 8, 14, 45, 44, 21, 31, 54, 51, 32, 50, 9, 13, 42, 22, 29, 48, 10, 12, 23, 11, 2\}
\end{itemize}


\subsection{Parallel Implementation}

The idea of parallel implementation is that all data - vertex coordinates, internal elements and boundary elements - are split between processes, not to exceed the single core memory. Thus the strategy for reading data on a process $i$ is as follows:
\begin{enumerate}
	\item Compute the total number $N_{elem}$ of non-boundary (internal) elements.
\end{enumerate}

\begin{mybox}
Loop over all elements in the file, and count the number of elements with dimension equal to the world dimension
\end{mybox}	
\begin{enumerate}[resume]	
	\item Read and store internal elements belonging to this process. If elements are split in consequent equal chunks among processes, then process $rank$ should read the elements with indices $interv(rank) = \floor[\Big]{ [rank, rank+1] \cdot N_{elem} / p_{tot} } + 1$.
\end{enumerate}

\begin{mybox}
\begin{itemize}
	\item Loop over all elements in the file
	\item Store all internal elements for which $index \in interv(rank)$
	\item Add global indices of vertices belonging to selected elements to a set
\end{itemize}

\end{mybox}		


\begin{enumerate}[resume]
	\item Read and store boundary elements belonging to this process - those which match the subentities of some elements.
\end{enumerate}
\begin{mybox}
	\begin{itemize}
		\item Only read the boundaries for which all \textbf{corners} have already been included to the element vertex set. 
	\end{itemize}
\end{mybox}	

\begin{enumerate}[resume]
	\item Read and store vertices belonging to this process. Namely, all the vertices that are necessary to construct the elements and boundaries belonging to 
this process.
	\begin{itemize}
		\item Local index of a vertex is a number $[0,n_p)$ where $n_p$ is the total number of vertices on the process.
		\item Local index of a vertex is given by the order they are inserted to the grid factory. It is in the ascending order of the global index, just that certain vertices of global index are not on this process. To keep track of this we fill the global-to-local map for vertices.
	\end{itemize}

	\item Add boundary elements to factory. It is necessary to connect the boundary elements to the internal elements they share a face with, because during load balance, the boundaries need to be communicated together with the corresponding elements.GMSH does not provide information on this interconnection.
	\begin{itemize}
		\item \uline{Important!} This must be done before adding internal elements to factory, as we also need to add the interconnection array.
		\item Add global element index as well
	\end{itemize}
\end{enumerate}

\begin{mybox}
\noindent
Currently using brute-force, because it is not much slower than improved for \\

\noindent
\uline{Trivial Algorithm: (Complexity $O(12 N_{elem} N_{\beta} / p_{tot}^2)$)}\\
\textit{Loop over all stored boundary elements $\beta_i$, and over all stored internal elements $E_j$.} \\
\textit{ If $\beta_i \in E_j$ for some $j$ then store $\beta_i$ }\\

\noindent
\uline{Improved Algorithm: (Complexity $O(12 N_{elem} \log_2 (N_{\beta} / p_{tot}) / p_{tot}$)}\\

\begin{enumerate}
	\item Construct map from boundary vertex index set to boundary id
	\item Add all boundaries to the map
	\item Loop over each face of all internal elements
	\begin{enumerate}
		\item If $map[face]$ is non-null, link the element and boundary
	\end{enumerate}
\end{enumerate}

\end{mybox}	
		
\begin{enumerate}[resume]
	\item Add internal elements to factory
\end{enumerate}
\begin{mybox}
	\begin{itemize}
		\item For debugging purposes write each element to a .vtk file using CurvilinearVTKWriter.
		\item Add element vertices and global element index to factory
		\item If creating grid with boundaries, also pass $internal\_to\_boundary\_element\_linker$. This array stores the indices of boundaries which are connected to this element (if any).
	\end{itemize}
\end{mybox}







\subsection{Partitioning}


%%%%%%%%%%%%%%%%%%%%%%%%%%%%%%%%%%%%%%%%%%%%%%%%%%%%%%%%%%%%%%%%%%%%%
% Implementation Details - Curvilinear Grid Constructor
%%%%%%%%%%%%%%%%%%%%%%%%%%%%%%%%%%%%%%%%%%%%%%%%%%%%%%%%%%%%%%%%%%%%%

\section{Implementation Details - Grid Constructor}
\label{impl-grid-constructor}



The Curvilinear Grid is constructed using \lstinline|Dune::CurvilinearGridConstructor|, which is wrapped by
\lstinline|CurvilinearGridFactory| $\rightarrow$
\lstinline|CurvilinearGrid| $\rightarrow$
\lstinline|CurvilinearGridBase| $\rightarrow$ \\
\noindent
\lstinline|CurvilinearGridConstructor|. The grid is constructed by first inserting all the vertices, elements and boundary segments, and then calling \lstinline|generateMesh()| as given in \ref{interface-grid-factory}. \\

\noindent
Afterwards, the construction of the grid goes according to following plan
\begin{itemize}
	\item Construction of entities - internal, process boundary and domain boundary edges and faces.
	\item Construction of the global index
	\item Construction of Ghost elements
	\item Construction of Iteration sets
	\item Construction of communication maps
	\item Construction of OCTree for hierarchical element location	
\end{itemize}


\subsection{Storage}

\noindent
The fundamental idea of the curvilinear grid is to store all of the information about the grid in a single class, namely \lstinline|CurvilinearGridStorage|. The following information is stored:

\begin{itemize}
	\item Number of entities over all processes, for each codim:  \lstinline|int nEntityTotal_[4]|
	\item Information about each entity stored in a vector, eg  \lstinline|std::vector<VertexStorage> point_;|. The index of the vector is the local index of the entity. Only elements store the indices of the corresponding interpolatory vertices, faces and edges do not. Instead, they store the index of the element they are the subentity of, and thus the corresponding vertices are extracted from the element when needed.

\begin{mybox}
\begin{lstlisting}
  Storage structures here
\end{lstlisting}
\end{mybox}	
	
	\item A vector of local indices of subentities of an element, as indexed by the local index of the element. Subentities are stored in the order corresponding to the way they are indexed in the \lstinline|Dune::ReferenceElement|
	
\begin{mybox}
\begin{lstlisting}
  std::vector< std::vector<LocalIndexType> > elementSubentityCodim1_;
  std::vector< std::vector<LocalIndexType> > elementSubentityCodim2_;	
\end{lstlisting}
\end{mybox}

	\item Local to global mapping is stored in a single map for all entities of each codimension.
	
\begin{mybox}
\begin{lstlisting}
  Global2LocalMap entityIndexMap_[4];
\end{lstlisting}
\end{mybox}

    \item For the purposes of iterating over certain entity types without having to iterate over all entities of a given codim, the below local index sets are stored:
        \subitem - All entities of a given codimension
        \subitem - Internal entities of a given codimension
        \subitem - Process boundary entities of a given codimension
        \subitem - Domain boundary entities of a given codimension
        \subitem - Ghost entities of a given codimension
        
        \subitem - Interior entities of a given codimension (As given in Dune - internal + domain boundaries)
        \subitem - Interior Border entities of a given codimension (As given in Dune - internal + domain boundaries + process boundaries)
        
\begin{mybox}
\begin{lstlisting}
  LocalIndexSet  entityAllIndexSet_[4];
  LocalIndexSet  entityInternalIndexSet_[4];
  LocalIndexSet  entityProcessBoundaryIndexSet_[4];
  LocalIndexSet  entityDomainBoundaryIndexSet_[4];
  LocalIndexSet  entityGhostIndexSet_[4];
    
  LocalIndexSet  entityDuneInteriorIndexSet_[4];
  LocalIndexSet  entityDuneInteriorBorderIndexSet_[4];
\end{lstlisting}
\end{mybox}

    \item For communication purposes, all entities that can communicate to other processes require their own local index. The below maps allow finding the associated special local index of the entity given its usual index:
        \subitem - boundary segments (faces only)
        \subitem - process boundary entities of all codimensions except 0, 
        \subitem - boundary internal entities of all codimensions (internal and domain boundary entities that neighbour the process boundaries )
        \subitem - ghost entities of all codimensions
        
\begin{mybox}
\begin{lstlisting}
    Local2LocalMap  boundarySegmentIndexMap_;
    Local2LocalMap  processBoundaryIndexMap_[4];
    Local2LocalMap  boundaryInternalEntityIndexMap_[4];
    Local2LocalMap  ghostIndexMap_[4];
\end{lstlisting}
\end{mybox}


    \item For communication purposes, all communicating entities need to store a vector of process ranks among which these entities are shared. These are further partitioned into possible communication protocols:
        \subitem - rocess Boundary $\rightarrow$ Process Boundary
        \subitem - Process Boundary $\rightarrow$ Ghost
        \subitem - Interior $\rightarrow$ Ghost
        \subitem - Ghost $\rightarrow$ Interior Border
        \subitem - Ghost $\rightarrow$ Ghost        

\begin{mybox}
\begin{lstlisting}
    EntityNeighborRankVector BI2GNeighborRank_[4];
    EntityNeighborRankVector PB2PBNeighborRank_[4];
    EntityNeighborRankVector PB2GNeighborRank_[4];
    EntityNeighborRankVector G2BIPBNeighborRank_[4];
    EntityNeighborRankVector G2GNeighborRank_[4];
\end{lstlisting}
\end{mybox}



\end{itemize}


\subsection{Entity construction}
\label{impl-grid-constructor-entity}

Entities are constructed in the following order
\begin{itemize}
	\item Insertion of vertices, elements and boundary segments by user. All inserted vertices and elements are marked as Internal. Boundary segments are marked as faces with StructuralType = DomainBoundary.
	\item Generation of edges as subentities of elements
	\item Generation of faces as subentities of elements. Faces with one neighbour become ProcessBoundaries, with two Internal.
	\item Vertices and edges that are subentities of Domain and Process Boundary faces are marked with same StructuralType.
	\item Generate unique local index for all Process Boundaries of each codim, to be further used for communication
\end{itemize}

Main ideas
\begin{itemize}
	\item All edges and faces store index of one parent element, as well as their internal index inside that element.
	\item Each element stores a list of local indices of all its subentity edges and faces in the order, given by Dune::ReferenceElement.
	\item Each element stores a list of local indices of all its interpolatory vertices as given in \ref{impl-gmsh-numbering-convention}
	\item Edges and faces do not store indices of associated vertices to save space. This information is obtained through parent element.
	\item All interpolatory vertices are considered subentities of corresponding elements, faces and edges. However, only entity corners possess unique process boundary index.
	\item For internal construction and interprocessor communication purposes, all entities are identified by associated keys. A key of an entity is a vector of global indices of all consisting corners, sorted in ascending order. Since the vertex global index is supplied by the user (GMSH), such keys are available for all entities at the beginning of grid construction.
\end{itemize}



\subsection{Global index construction}
\label{impl-grid-constructor-globalindex}

\noindent
The challenge in computing the global index comes from the fact that originally each process does not have any information about other processes, most importantly that it does not know its neighbours. Therefore, this necessary information is also computed. The algorithm is as follows

\begin{enumerate}
	\item Communicate neighbour ranks associated with each Process Boundary (PB) corner. Currently, all processes communicate all their process boundary corner global indices to all other processes. Thus, from the received global indices, each process can deduce all other processes sharing each of its corners. This algorithm is inefficient and does not scale with high process number. The ultimate hope would be to find this information only through nearest neighbour communication, however, the algorithm is not obvious. The reason to expect that a more optimal algorithm is available is because the following considerations are not used
	  \subitem - for most PB points, there should be exactly 1 neighbor, all other cases are progressively more rare
	  \subitem - for most PB points, the surrounding points are very likely to be shared by the same processes
	\item Compute (provisional) neighbour ranks of PB edges and faces by intersection of ranks of associated PB corners
      \subitem - Sometimes, an entity does not exist on a neighbouring process, even though all associated PB corners are present. This only happens if the (provisional) number of neighbors is larger than 1 (complicated PB entity), because each PB entity must have at least 1 neighbour.
       \subitem - For each complicated PB entity (edge/face), communicate EdgeKeys and FaceKeys to all provisional neighbours
       \subitem - For each received key, reply to sender if such entity exists on this process or not
       \subitem - Remove neighbour ranks mapping to non-existing entities
    \item For each PB edge and face, determine if it is owned by this process. A shared entity is owned by the process with lowest rank. Communicate the number of elements, edges and faces owned by each process to all other processes. Note that each process owns all its elements since they are not shared.
    \item Locally enumerate all edges, faces and elements owned by this process. Each process starts enumeration of global index of owned entities of a particular codim with the total number of entities of that codim owned by all processes with rank lower than this process. It then enumerates all its owned entities with consecutive integers. 
    \item Communicate globalIndices of entities owned by this process to all other processes sharing it.
       \subitem - By analysing entity neighbours, each process can compute how many how many global indices it needs to send and to receive to each other process
       \subitem - Each process sends to each neighbour the shared entity global indices enumerated by this process and receives those enumerated by the neighbour process
    \item Fill in Global2Local maps. They are required for user functionality and for construction of GhostElements

	
\end{enumerate}

     * [TODO] MinRank-Ownership paradigm non-uniform. If ever becomes bottleneck, replace by XORRank-Ownership








\subsection{Ghost element construction}
\label{impl-grid-constructor-ghost}

\noindent
Ghost entities are the subentities of the element on the other side of the process boundary face, including the element itself. The process boundary entities are not considered ghost entities. Thus, the ghost entities are the internal/domain boundary entities of another process, borrowed by this process. Construction of ghost entities involves communicating all the information associated with the entities to the neighbouring processes, and then incorporating the ghost elements into the grid on the receiving side. \\

\noindent
Since for every PB face the neighbouring process rank has already been determined in the previous section, and the global index is already known, there is nothing fundamentally tricky about this part of the grid construction. It is just very tedious. The algorithm is given below:


\begin{enumerate}
	\item Compute number of process boundaries shared with each process, as well as and the exact interpolation orders of Ghost Elements
		\subitem - Note that a ghost element can have more than 1 process boundary face associated to it, so a set of faces needs to be stored
		\subitem - Note that for this reason it is not possible to know in advance how many ghosts will be received from a neighbor, so it has to be communicated
	\item All2AllCommunication - communicate to each process a list of interpolation orders of the Ghost Elements it is going to receive
	\item All2AllCommunication - Package and send element global index, physical tag, all subentity global indices, all internal indices of PB faces of this element
	\item Add all ghost elements to the mesh. Calculate which vertices are missing from which processes
	\item All2AllCommunication - Communicates to each process the number of missing vertices out of the ones it had communicated.
		\subitem - Communicate the globalIndices's of all missing vertices
		\subitem - Communicate the vertex coordinates corresponding to received global indices
	\item Distrubute vertex coordinates and add received coordinates to the mesh	
\end{enumerate}

\noindent
Notes: 

\begin{itemize}
	\item Only supports tetrahedral ghost elements at the moment
	\item Ghost elements can have different interpolation order 
\end{itemize}





\subsection{Communication interface construction}
\label{impl-grid-constructor-comm}

Communication implies that all entities would send some information to their copies on other processors. Depending on the communication protocol, only entities of a particular structural type will communicate on the sending and receiving end. In this section we will use shorthands
\begin{itemize}
	\item PB - Process boundaries, that is, process boundary faces and their subentities
	\item I - Internal elements, which neighbour a process boundary, and subentities of such elements, including domain boundaries but excluding the process boundary entities.
	\item G - Ghost elements and their subentities, excluding process boundaries
\end{itemize}

\noindent
The below table presents the communication interfaces used by Dune, and explains how it translates to communication between entities of above specified structural types \\

\noindent
\begin{tabular}{ | l | l | l | l | l | l | l | l | }
  \hline
  Interface & Direction &
      \lstinline|PB_PB| &
      \lstinline|PB_G| &
      \lstinline|I_G| &
      \lstinline|G_I| &
      \lstinline|G_PB| &
      \lstinline|G_G| \\ \hline
  \lstinline|InteriorBorder_InteriorBorder| & ---      & Y & N & N & N & N & N \\ \hline
  \lstinline|InteriorBorder_All|            & Forward  & Y & Y & Y & N & N & N \\ \hline
  \lstinline|InteriorBorder_All|            & Backward & Y & N & N & Y & Y & N \\ \hline
  \lstinline|All_All|                       & ---      & Y & Y & Y & Y & Y & Y \\ \hline
\end{tabular} \\

\noindent
The aim of this constructor is to generate unique index maps for the sets PB, I and G (the set PB already has a unique index map, it will be reused). Then, for every communicating entity, for every possible structural type pair, we require an array of ranks of the processes with which such communication is possible. Note that the map for protocol $PB\rightarrow PB$ already exists, it is constructed in the very beginning of grid construction to enable construction of global indices and ghost elements. The algorithm is as follows

\begin{enumerate}
	\item Create unique index for sets I and G, by iterating over PB faces and adding corresponding subentities of neighbouring elements. Also, mark the neighbour rank of the associated PB for all I and G entities visited this way, thus enabling the protocols $I \rightarrow G$ and $G \rightarrow I$. Note that entities of all (!!) codimensions can have more than one neighbour rank obtained this way.
		\subitem - During the marking, may encounter elements with two or more process boundaries from different processes. In that case, for each process boundary entity we mark the rank of the other process boundary, thus providing some information for the future construction of the protocol $PB \rightarrow G$.
	\item After the above, all entities that can be communicated over should be associated an array of ranks of all other processes over which this entity is shared. What remains is to finish the protocol $PB \rightarrow G$, and calculate the remaining protocols $G \rightarrow I$, $G \rightarrow PB$ and $G \rightarrow G$.
	\item Iterate over all PB entities
		\subitem divide provisional $PB \rightarrow G$ set by $PB\rightarrow PB$ set to ensure that the same process has not been added by accident
		\subitem Mark number of real $PB \rightarrow G$ candidates for each process
	\item For all PB entities having non-zero $PB \rightarrow G$, communicate G to all neighbouring PB
	\item For all PB append received G by using union on them - This completes $PB \rightarrow G$ (hopefully)
	\item For all PB entities having non-zero $PB \rightarrow G$, communicate self to all G of ($PB \rightarrow G$)
	\item For all G append received PB by using union on them - This completes $G \rightarrow PB$  (hopefully)
	\item For all PB entities having non-zero $PB \rightarrow G$, communicate to all G all remaining G
		\subitem Optimization - do this only if you are lowest rank among all PB-neighbors
		\subitem Further optimization - do this only if you are modulus-rank among all PB-neighbors
	\item For all G append received G by using union on them - This completes $G \rightarrow G$ (hopefully)	
\end{enumerate}
    







\subsection{Iteration set construction}
\label{impl-grid-constructor-iterator}

Construction of the iterator sets involves simply iterating over all entities, and filling the sets with local indices based on the entity structural type.


\subsection{OCTree construction}
\label{impl-grid-constructor-octree}
































%%%%%%%%%%%%%%%%%%%%%%%%%%%%%%%%%%%%%%%%%%%%%%%%%%%%%%%%%%%%%%%%%%%%%
% Implementation Details - Curvilinear VTK Writer
%%%%%%%%%%%%%%%%%%%%%%%%%%%%%%%%%%%%%%%%%%%%%%%%%%%%%%%%%%%%%%%%%%%%%

\include{manual-vtkwriter}







\appendix

\chapter{Proofs}

%%%%%%%%%%%%%%%%%%%%%%%%%%%%%%%%%%%%%%%
% Proof of polynomial integral over simplex
%%%%%%%%%%%%%%%%%%%%%%%%%%%%%%%%%%%%%%%
\include{manual-geometry-appendix-polynomialintegralproof}









%%%%%%%%%%%%%%%%%%%%%%%%%%%%%%%%%%%%%%%
% Bibliography
%%%%%%%%%%%%%%%%%%%%%%%%%%%%%%%%%%%%%%%


%\bibliographystyle{plainnat}
%\bibliography{oswald,numerical_libraries,molecular_plasmonics,finite_element_method,electromagnetic}


\printindex

\end{document}